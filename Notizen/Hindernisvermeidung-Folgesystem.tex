% ju 19-Feb-24 Hindernisvermeidung-Folgesystem.tex
\documentclass{vorlage-design-main}
\usepackage[utf8]{inputenc}
\usepackage{longtable}
\usepackage{blindtext,alltt}
%% Ganze Überschrift
\title{Thema}

%% Kürzerer Titel zur Verwendung im Seitenkopf
\runningtitle{Kurztitel}
\author{Jan Unger}
% \author{2.}
\date{\today}

%% Die .bib-Datei mit vollständigen Referenzen zur Verwendung mit biblatex. articleclass lädt das Paket biblatex-chicago mit Anpassungen
\addbibresource{literatur.bib}

\begin{document}

\maketitle

\begin{abstract}

\end{abstract}

\hypertarget{fortgeschrittene-hindernisvermeidung-und-intelligentes-folgesystem}{%
\subsection{Fortgeschrittene Hindernisvermeidung und intelligentes
Folgesystem}\label{fortgeschrittene-hindernisvermeidung-und-intelligentes-folgesystem}}

\begin{itemize}

\item
  \textbf{Einleitung und Zielsetzung}

  \begin{itemize}
  
  \item
    Entwicklung von STEAM-Fähigkeiten durch Kombination eines
    Hindernisvermeidungsmoduls mit einem Ultraschallsensor.
  \item
    Implementierung eines intelligenten Folgesystems für einen
    Mars-Rover, ermöglicht Hindernisvermeidung und Folgen von
    beweglichen Objekten.
  \end{itemize}
\item
  \textbf{Verständnis der Konzepte}:

  \begin{itemize}
  
  \item
    Hindernisvermeidung durch Infrarotsensoren, die reflektierte
    Infrarotsignale von Objekten erfassen.
  \item
    Ultraschallsensoren messen Entfernungen durch Aussenden und
    Empfangen von Schallwellen, um die Distanz zu Objekten zu bestimmen.
  \item
    Kombination beider Systeme für ein effizientes
    Hindernisvermeidungssystem.
  \end{itemize}
\end{itemize}

\hypertarget{erstellung-eines-fortgeschrittenen-hindernisvermeidungssystems}{%
\subsection{Erstellung eines fortgeschrittenen
Hindernisvermeidungssystems}\label{erstellung-eines-fortgeschrittenen-hindernisvermeidungssystems}}

\hypertarget{keywords-fortgeschrittenen-hindernisvermeidungssystems}{%
\subsubsection{Keywords fortgeschrittenen
Hindernisvermeidungssystems}\label{keywords-fortgeschrittenen-hindernisvermeidungssystems}}

\begin{itemize}
\item
  \textbf{SoftPWM}: Eine Bibliothek, die softwarebasierte
  Pulsweitenmodulation (PWM) auf Arduino-Pins ermöglicht. Dies wird
  verwendet, um die Geschwindigkeit der Motoren zu steuern.
\item
  \textbf{ULTRASCHALLSENSOR\_PIN}: Der Pin, an den der Ultraschallsensor
  angeschlossen ist. Dieser Sensor misst die Entfernung zu Hindernissen.
\item
  \textbf{IR\_RIGHT, IR\_LEFT}: Pins, an die die Infrarot(IR)-Sensoren
  angeschlossen sind. Diese Sensoren erkennen Hindernisse auf der
  rechten bzw. linken Seite des Rovers.
\item
  \textbf{LEFT\_MOTOR\_FORWARD\_PIN, LEFT\_MOTOR\_REVERSE\_PIN}: Pins,
  die die Vorwärts- und Rückwärtsbewegung der linken Motoren steuern.
\item
  \textbf{RIGHT\_MOTOR\_FORWARD\_PIN, RIGHT\_MOTOR\_REVERSE\_PIN}: Pins,
  die die Vorwärts- und Rückwärtsbewegung der rechten Motoren steuern.
\item
  \textbf{Serial.begin(115200)}: Startet die serielle Kommunikation mit
  einer Baudrate von 115200. Dies wird für Debugging-Zwecke verwendet.
\item
  \textbf{digitalRead()}: Liest den Wert (HIGH oder LOW) eines
  spezifizierten digitalen Pins.
\item
  \textbf{SoftPWMSet(pin, speed)}: Steuert die Geschwindigkeit eines
  Motors, indem es die PWM-Signalstärke auf einem bestimmten Pin setzt.
\item
  \textbf{delay()}, \textbf{delayMicroseconds()}: Pausiert das Programm
  für eine bestimmte Zeit. \verb|delay()| verwendet
  Millisekunden, während \verb|delayMicroseconds()|
  Mikrosekunden verwendet.
\item
  \textbf{pinMode()}: Konfiguriert den angegebenen Pin, um entweder als
  Eingang (INPUT) oder als Ausgang (OUTPUT) zu fungieren.
\item
  \textbf{digitalWrite()}: Setzt einen digitalen Pin auf HIGH (an) oder
  LOW (aus).
\item
  \textbf{pulseIn()}: Misst die Dauer eines Pulses auf einem Pin. Wird
  verwendet, um die Echo-Zeit vom Ultraschallsensor zu messen.
\item
  \textbf{moveForward()}, \textbf{moveBackward()}, \textbf{backRight()},
  \textbf{backLeft()}: Funktionen, die den Rover in verschiedene
  Richtungen steuern, basierend auf den Eingaben von den Sensoren.
\item
  \textbf{readSensorData()}: Liest die Entfernungswerte vom
  Ultraschallsensor und berechnet die Distanz zu einem Objekt basierend
  auf der Zeitdauer des Echo-Signals.
\end{itemize}

\hypertarget{initialisierung}{%
\subsubsection{Initialisierung}\label{initialisierung}}

\begin{itemize}

\item
  Das Programm beginnt mit der \textbf{Initialisierung} der Sensoren und
  Motoren im \verb|setup()|-Teil. Hier werden die
  Pins für die Ultraschall- und Infrarotsensoren sowie für die
  Motorsteuerung konfiguriert. Die serielle Kommunikation wird ebenfalls
  gestartet, um Debugging-Informationen zu übertragen.
\end{itemize}

\hypertarget{hauptlogik-im-loop}{%
\subsubsection{Hauptlogik im loop()}\label{hauptlogik-im-loop}}

\begin{itemize}

\item
  Im Hauptteil des Programms, innerhalb der
  \verb|loop()|-Funktion, erfolgt die kontinuierliche
  Überwachung der \textbf{Infrarotsensoren}. Diese Sensoren detektieren,
  ob Hindernisse auf der rechten oder linken Seite des Rovers vorhanden
  sind.
\item
  Basierend auf den Eingaben der Infrarotsensoren, trifft das Programm
\end{itemize}

\textbf{Entscheidungen}:

\begin{itemize}

\item
  Wenn das \textbf{rechte IR-Modul blockiert} ist, ruft das Programm
  \verb|backRight()| auf, um den Rover nach rechts zu
  drehen.
\item
  Wenn das \textbf{linke IR-Modul blockiert} ist, führt es
  \verb|backLeft()| aus, um nach links zu drehen.
\item
  Bei \textbf{Blockierung beider IR-Module} aktiviert es
  \verb|moveBackward()|, um den Rover rückwärts zu
  bewegen.
\item
  Ist \textbf{kein Hindernis} im Weg, wird
  \verb|handleForwardMovement()| aufgerufen.
\end{itemize}

\hypertarget{distanzmessung-und-bewegungssteuerung}{%
\subsubsection{Distanzmessung und
Bewegungssteuerung}\label{distanzmessung-und-bewegungssteuerung}}

\begin{itemize}

\item
  Die Funktion \verb|handleForwardMovement()| nutzt
  den \textbf{Ultraschallsensor}, um die Distanz zum nächsten Hindernis
  zu messen (mittels \verb|readSensorData()|).
  Abhängig von dieser Distanz entscheidet das Programm über die nächste
  Aktion:

  \begin{itemize}
  
  \item
    Bei einer \textbf{Distanz über 30 cm} wird
    \verb|moveForward()| mit höherer Geschwindigkeit
    aufgerufen, da der Weg als sicher gilt.
  \item
    Bei einer \textbf{Distanz zwischen 2 cm und 15 cm} deutet die Nähe
    eines Hindernisses darauf hin, dass zunächst
    \verb|moveBackward()| und dann
    \verb|backLeft()| ausgeführt wird, um eine
    Kollision zu vermeiden und eine neue Richtung einzuschlagen.
  \item
    Bei \textbf{Distanzen zwischen 15 cm und 30 cm} führt der Rover eine
    vorsichtige Vorwärtsbewegung durch, indem
    \verb|moveForward()| mit reduzierter
    Geschwindigkeit aufgerufen wird.
  \end{itemize}
\end{itemize}

\hypertarget{datenfluss-und-fehlerbehandlung}{%
\subsubsection{Datenfluss und
Fehlerbehandlung}\label{datenfluss-und-fehlerbehandlung}}

\begin{itemize}

\item
  Der \textbf{Datenfluss} im Programm ist strukturiert, um
  kontinuierlich Sensorwerte zu lesen und darauf basierend
  Entscheidungen zu treffen. Die Funktion
  \verb|readSensorData()| spielt eine zentrale Rolle,
  indem sie präzise Distanzmessungen liefert, die für
  Bewegungsentscheidungen entscheidend sind.
\item
  \textbf{Fehlerbehandlung} kann in Form von Überprüfungen der
  Sensorwerte implementiert sein, z.B. wenn ungültige (zu hohe oder zu
  niedrige) Distanzwerte gelesen werden, was auf ein Problem mit dem
  Sensor hinweisen könnte. Diese Logik könnte innerhalb der
  \verb|readSensorData()|-Funktion oder nach dem
  Aufruf dieser Funktion zur Anwendung kommen, ist aber im
  bereitgestellten Code nicht explizit enthalten.
\end{itemize}

\hypertarget{benutzerinteraktion}{%
\subsubsection{Benutzerinteraktion}\label{benutzerinteraktion}}

\begin{itemize}

\item
  \textbf{Benutzerinteraktion} ist nicht direkt im Code enthalten, aber
  die serielle Ausgabe von Distanzwerten und möglicherweise von
  Debugging-Informationen ermöglicht eine Beobachtung und Interaktion
  während der Laufzeit.
\end{itemize}

\hypertarget{schleifen-und-bedingte-anweisungen}{%
\subsubsection{Schleifen und bedingte
Anweisungen}\label{schleifen-und-bedingte-anweisungen}}

\begin{itemize}

\item
  Das Programm nutzt \textbf{Schleifen} (repräsentiert durch die
  ständige Wiederholung der \verb|loop()|-Funktion)
  und \textbf{bedingte Anweisungen} (if-else-Konstrukte), um auf Basis
  der Sensorwerte Entscheidungen zu treffen und entsprechende Aktionen
  auszuführen.
\end{itemize}

\newpage

\begin{lstlisting}
                  +-----------------+
                  |     Start       |
                  +-----------------+
                            |
                            v
               +----------------------------+
               | Initialisiere Systeme      |
               | und Sensoren               |
               +----------------------------+
                            |
                            v
               +----------------------------+
               | Lese IR-Sensorwerte        |
               +----------------------------+
                            |
                            v
            +---------------+----------------+
            |                                |
    +-------v-------+                +-------v-------+
    | Rechtes IR    |                | Linkes IR     |
    | blockiert?    |                | blockiert?    |
    +---------------+                +---------------+
     |       |                             |       |
    Ja      Nein                          Ja      Nein
     |       |                             |       |
     v       |                             v       |
+----+----+  |                      +----+----+    |
|backRight|  |                      | backLeft|    |
+---------+  |                      +---------+    |
             |                                     |
             +----------------+--------------------+
                              |
                              v
                   +----------+-----------+
                   | Beide IR blockiert?  |
                   +----------+-----------+
                           |         |
                          Ja        Nein
                           |         |
                           v         v
               +-----------+--+  +---+------------+
               | moveBackward |  | handleForward  |
               +--------------+  +----------------+
                                          |
                                          v
                              +-----------+------------+
                              | Lese Ultraschalldistanz|
                              +-----------+------------+
                                          |
                                          v
                             +------------+-----------+
                             | Entscheide Bewegung    |
                             | basierend auf Distanz  |
                             +------------+-----------+
                                          |
                  +-----------------------+-------------------------+
                  |                       |                         |
                  v                       v                         v
      +-----------+-----------+ +---------+----------+  +-----------+-------------+
      | moveForward (sicher)  | | moveBackward       |  | moveForward (vorsichtig)|
      | wenn > 30cm           | | und backLeft       |  | wenn < 30cm & > 15cm    |
      |                       | | wenn < 15cm & > 2cm|  +-------------------------+
      +-----------------------+ +--------------------+
\end{lstlisting}

\newpage

\begin{lstlisting}[language={C++}]
/**
 * @file main.cpp
 * @brief Erstellung eines fortgeschrittenen Hindernisvermeidungssystems
 * 
 * durch Integration von Ultraschall- und Infrarotsensoren.
 */
#include <SoftPWM.h>

// Definiere den Pin für das Ultraschallmodul
#define ULTRASCHALLSENSOR_PIN 10

// Definiere die Pins für die IR-Module
#define IR_RIGHT 7
#define IR_LEFT 8

// Definition der Pins für die linken Motoren A, B, C
#define LEFT_MOTOR_FORWARD_PIN 2 // Pin für Vorwärtsbewegung der linken Motoren (A, B, C)
#define LEFT_MOTOR_REVERSE_PIN 3 // Pin für Rückwärtsbewegung der linken Motoren (A, B, C)

// Definition der Pins für die rechten Motoren D, E, F
#define RIGHT_MOTOR_FORWARD_PIN 5 // Pin für Vorwärtsbewegung der rechten Motoren (D, E, F)
#define RIGHT_MOTOR_REVERSE_PIN 4 // Pin für Rückwärtsbewegung der rechten Motoren (D, E, F)

void setup() {

  // Initialisiere die serielle Kommunikation zur Fehlersuche
  Serial.begin(115200);

  // Initialisiere SoftPWM
  SoftPWMBegin();

  // Setze die Pins der IR-Module als Eingänge
  pinMode(IR_RIGHT, INPUT);
  pinMode(IR_LEFT, INPUT);
}

void loop() {
  // Lese Werte von den IR-Sensoren
  int rightValue = digitalRead(IR_RIGHT);
  int leftValue = digitalRead(IR_LEFT);

  // Steuere die Bewegungen des Rovers basierend auf den Lesungen der IR-Sensoren
  if (rightValue == 0 && leftValue == 1) {  // Rechtes IR-Modul blockiert
    backRight(150);
  } else if (rightValue == 1 && leftValue == 0) {  // Linkes IR-Modul blockiert
    backLeft(150);
  } else if (rightValue == 0 && leftValue == 0) {  // Beide Module blockiert
    moveBackward(150);
  } else {  // Weg frei
    handleForwardMovement();
  }
}

void handleForwardMovement() {
  // Lese die Distanz vom Ultraschallsensor
  float distance = readSensorData();
  //Serial.println(distance);  // Ausgabe der Distanz zur Fehlersuche

  // Steuere den Rover basierend auf der Distanzmessung
  if (distance > 30) {  // Wenn es sicher ist, vorwärts zu bewegen
    moveForward(200);
  } else if (distance < 15 && distance > 2) {  // Wenn ein Hindernis nahe ist
    moveBackward(200);
    delay(500);  // Warte kurz, bevor versucht wird, abzubiegen
    backLeft(150);
    delay(1000);
  } else {  // Für Zwischendistanzen mit Vorsicht verfahren
    moveForward(150);
  }
}

float readSensorData() {
  // Eine Verzögerung von 4ms ist erforderlich, sonst könnte die Messung 0 sein
  delay(4);

  // Pin auf OUTPUT setzen, um Signal zu senden
  pinMode(ULTRASCHALLSENSOR_PIN, OUTPUT);

  // Den Trigger-Pin zurücksetzen
  digitalWrite(ULTRASCHALLSENSOR_PIN, LOW);
  delayMicroseconds(2);

  // Den Sensor durch Senden eines hohen Pulses für 10us auslösen
  digitalWrite(ULTRASCHALLSENSOR_PIN, HIGH);
  delayMicroseconds(10);
  digitalWrite(ULTRASCHALLSENSOR_PIN, LOW);

  // Pin auf INPUT setzen, um zu lesen
  pinMode(ULTRASCHALLSENSOR_PIN, INPUT);

  // pulseIn gibt die Dauer des Pulses am Pin zurück
  float duration = pulseIn(ULTRASCHALLSENSOR_PIN, HIGH);

  // Berechne die Distanz (in cm) basierend auf der Schallgeschwindigkeit (340 m/s oder 0.034 cm/us)
  float distance = duration * 0.034 / 2;

  return distance;
}

// Funktion, um den Rover vorwärts zu bewegen
void moveForward(int speed) {
  // Linke Motoren gegen den Uhrzeigersinn rotieren lassen
  SoftPWMSet(LEFT_MOTOR_FORWARD_PIN, speed);
  SoftPWMSet(LEFT_MOTOR_REVERSE_PIN, 0);

  // Rechte Motoren im Uhrzeigersinn rotieren lassen
  SoftPWMSet(RIGHT_MOTOR_FORWARD_PIN, 0);
  SoftPWMSet(RIGHT_MOTOR_REVERSE_PIN, speed);
}

// Funktion, um den Rover rückwärts zu bewegen
void moveBackward(int speed) {
  // Linke Motoren im Uhrzeigersinn rotieren lassen
  SoftPWMSet(LEFT_MOTOR_FORWARD_PIN, 0);
  SoftPWMSet(LEFT_MOTOR_REVERSE_PIN, speed);

  // Rechte Motoren gegen den Uhrzeigersinn rotieren lassen
  SoftPWMSet(RIGHT_MOTOR_FORWARD_PIN, speed);
  SoftPWMSet(RIGHT_MOTOR_REVERSE_PIN, 0);
}

// Funktion, um nach hinten rechts abzubiegen
void backRight(int speed) {
  SoftPWMSet(LEFT_MOTOR_FORWARD_PIN, 0);
  SoftPWMSet(LEFT_MOTOR_REVERSE_PIN, speed);
  SoftPWMSet(RIGHT_MOTOR_FORWARD_PIN, 0);
  SoftPWMSet(RIGHT_MOTOR_REVERSE_PIN, 0);
}

// Funktion, um nach hinten links abzubiegen
void backLeft(int speed) {
  SoftPWMSet(LEFT_MOTOR_FORWARD_PIN, 0);
  SoftPWMSet(LEFT_MOTOR_REVERSE_PIN, 0);
  SoftPWMSet(RIGHT_MOTOR_FORWARD_PIN, speed);
  SoftPWMSet(RIGHT_MOTOR_REVERSE_PIN, 0);
}
\end{lstlisting}

\newpage

\hypertarget{schluxfcsselbegriffe-und-bedeutungen}{%
\subsubsection{Schlüsselbegriffe und
Bedeutungen}\label{schluesselbegriffe-und-bedeutungen}}

\begin{itemize}
\item
  \textbf{Ultraschallsensor (ULTRASCHALLSENSOR\_PIN):} Ein Sensor, der
  Ultraschallwellen aussendet und reflektierte Signale empfängt, um die
  Entfernung zu Objekten zu messen. Dieser Sensor ist zentral für die
  Hinderniserkennung und -vermeidung.
\item
  \textbf{Infrarotsensoren (IR\_RIGHT, IR\_LEFT):} Sensoren, die
  Infrarotlicht aussenden und empfangen, um Hindernisse auf der rechten
  und linken Seite des Rovers zu erkennen. In diesem Code-Beispiel
  werden die Pins definiert, aber die IR-Sensoren selbst werden nicht
  aktiv verwendet.
\item
  \textbf{Motor-Pins (LEFT\_MOTOR\_FORWARD\_PIN, etc.):} Spezifische
  Pins, die mit den Motoren des Rovers verbunden sind, um Vorwärts-,
  Rückwärts- und Drehbewegungen zu steuern.
\item
  \textbf{Zustandsvariable (State):} Eine Enumeration, die verschiedene
  Zustände des Rovers repräsentiert, wie z.B. Vorwärtsbewegung,
  Hinderniserkennung, Rückwärtsbewegung usw. Sie steuert die Logik der
  Bewegungen basierend auf den Sensordaten.
\item
  \textbf{Geschwindigkeitssteuerung (FORWARD\_SPEED, CAUTIOUS\_SPEED,
  BACKWARD\_SPEED):} Variablen, die die Geschwindigkeit des Rovers in
  verschiedenen Szenarien definieren, um eine effektive Navigation und
  Hindernisvermeidung zu gewährleisten.
\item
  \textbf{Verzögerungen (backwardDelay, turnDelay):} Zeitintervalle, die
  steuern, wie lange der Rover in einem bestimmten Zustand verbleibt,
  bevor er zum nächsten Zustand wechselt. Diese Verzögerungen helfen,
  abrupte Bewegungsänderungen zu vermeiden.
\item
  \textbf{Sensorleseintervall (sensorReadInterval):} Das Zeitintervall
  zwischen zwei Lesevorgängen des Ultraschallsensors, um die Distanz zu
  Hindernissen kontinuierlich zu aktualisieren.
\item
  \textbf{Geschwindigkeitsanpassung (adjustSpeed):} Eine Funktion, die
  schrittweise die Geschwindigkeit der Motoren anpasst, um eine sanftere
  Beschleunigung oder Verzögerung zu ermöglichen.
\item
  \textbf{Bewegungsfunktionen (moveForward, moveBackward, backLeft,
  etc.):} Spezifische Funktionen, die die Bewegungen des Rovers steuern,
  basierend auf den Entscheidungen der Zustandsmaschine und den
  Sensordaten.
\end{itemize}

Ultraschallsensor, Infrarotsensoren, State, Geschwindigkeitssteuerung,
Verzögerungen, Sensorleseintervall, Geschwindigkeitsanpassung,
Bewegungsfunktionen

\newpage

\begin{lstlisting}
                 +-------------------+
                 |    Start          |
                 +-------------------+
                         |
                         v
              +------------------------+
              | Initialisiere Systeme  |
              | und Sensoren           |
              +------------------------+
                         |
                         v
                 +-------------------+
                 | Loop beginnt      |
                 +-------------------+
                         |
  +----------------------+-----------------------+
  |                      |                       |
  v                      v                       v
+----------------+  +-------------+        +-----------------+
| Sensorlese-    |  | Prüfe, ob   |        | Handle          |
| intervall      |  | Distanz ==  |        | Fehlerzustand   |
| abgelaufen?    |  | -1.0        |        |                 |
+----------------+  +-------------+        +-----------------+
  | Ja               | Ja             | Nein              | Ja
  |                  |                |                   |
  v                  v                v                   v
+----------------+  +-------------+  +------------+  +--------------+
| Lese Sensor-   |  | Stoppe       | | Update     |  | Stoppe       |
| daten          |  | Bewegung &   | | Zustand    |  | Bewegung,    |
|                |  | zeige Fehler | | basierend  |  | zeige        |
+----------------+  +-------------+  | auf Sensor-|  | Fehlermeldung|
                                     | daten      |  +--------------+
                                     +------------+
                                                  |
                                    +-------------+---------------+
                                    |                             |
                                    v                             v
                      +-------------------------+ +-------------------------+
                      | FORWARD:                | | BACKWARD, TURN_LEFT:    |
                      | Bewege vorwärts         | | Führe nicht-vorwärts    |
                      | basierend auf           | | Bewegungen aus          |
                      | Geschwindigkeits-       | | und aktualisiere        |
                      | anpassung und           | | den Zustand entsprechend|
                      | Distanzdaten            | +-------------------------+
                      +-------------------------+
\end{lstlisting}

\newpage

\hypertarget{beschreibung}{%
\subsubsection{Beschreibung}\label{beschreibung}}

Das Programm, das Sie beschreiben, steuert einen autonomen Rover durch
die Integration verschiedener Sensoren und Bewegungslogiken. Der Kern
des Programms basiert auf einem zyklischen Ablauf (Loop), in dem
kontinuierlich Daten von Ultraschallsensoren und Infrarotsensoren
gelesen werden, um die Umgebung des Rovers zu beurteilen und
entsprechend zu navigieren. Hier eine detaillierte Beschreibung der
Hauptlogik und des Datenflusses:

\hypertarget{initialisierung-1}{%
\paragraph{Initialisierung}\label{initialisierung-1}}

Zu Beginn initialisiert das Programm die Systeme und Sensoren. Dies
umfasst die Konfiguration der Pin-Belegungen für den Ultraschallsensor
und die Infrarotsensoren sowie die Initialisierung der Motorsteuerung
über die SoftPWM-Bibliothek. Die serielle Kommunikation wird ebenfalls
gestartet, um Ausgaben für Debugging-Zwecke zu ermöglichen.

\hypertarget{hauptlogik-loop}{%
\paragraph{Hauptlogik (Loop)}\label{hauptlogik-loop}}

Im Hauptteil des Programms findet die wiederkehrende Logik statt, die
sich wie folgt zusammensetzt:

\begin{enumerate}
\def\labelenumi{\arabic{enumi}.}
\item
  \textbf{Sensorabfrage:} Das Programm prüft regelmäßig, ob das
  Sensorleseintervall abgelaufen ist, und liest dann die aktuellen
  Distanzwerte von den Sensoren. Diese Daten werden verwendet, um die
  Umgebung des Rovers zu verstehen und Hindernisse zu erkennen.
\item
  \textbf{Fehlerbehandlung:} Wenn eine Sensorleseoperation fehlschlägt
  (erkennbar an einem Rückgabewert von -1.0 für die Distanzmessung),
  tritt das Programm in einen Fehlerzustand ein. In diesem Fall stoppt
  der Rover seine Bewegung, und es wird eine Fehlermeldung ausgegeben.
\item
  \textbf{Entscheidungsfindung und Zustandsmanagement:} Basierend auf
  den gelesenen Sensordaten entscheidet das Programm über den nächsten
  Zustand des Rovers. Zustände wie \verb|FORWARD|,
  \verb|BACKWARD|,
  \verb|TURN\_LEFT| etc. steuern die
  Bewegungsrichtung des Rovers. Jeder Zustand ist mit bestimmten
  Aktionen verbunden, wie z.B. Vorwärtsbewegen, wenn der Weg frei ist,
  oder Umkehren und Drehen bei erkannten Hindernissen.
\item
  \textbf{Geschwindigkeitssteuerung und Bewegungsfunktionen:} Die
  Geschwindigkeit des Rovers wird dynamisch angepasst, basierend auf der
  Distanz zu Hindernissen und dem aktuellen Bewegungszustand. Die
  Bewegungsfunktionen wie \verb|moveForward()|,
  \verb|moveBackward()|,
  \verb|turnLeft()|,
  \verb|turnRight()| und
  \verb|stopMove()| werden entsprechend aufgerufen,
  um die Motoren des Rovers anzusteuern.
\item
  \textbf{Geschwindigkeitsanpassung:} Um sanfte Bewegungsübergänge zu
  gewährleisten, verwendet das Programm eine schrittweise Anpassung der
  Geschwindigkeit. Dies hilft dabei, abrupte Bewegungen zu vermeiden und
  die Mechanik des Rovers zu schonen.
\end{enumerate}

\hypertarget{schluxfcsselaspekte}{%
\paragraph{Schlüsselaspekte}\label{schluesselaspekte}}

\begin{itemize}

\item
  \textbf{Datenfluss:} Die Sensordaten fließen von den Sensoren zum
  Entscheidungssystem des Programms, welches basierend auf diesen Daten
  und dem aktuellen Zustand des Rovers Entscheidungen trifft.
\item
  \textbf{Loop und bedingte Anweisungen:} Das Herzstück des Programms
  bildet eine Loop-Struktur, innerhalb derer bedingte Anweisungen
  (if-else-Strukturen und switch-cases) genutzt werden, um auf Basis der
  Sensordaten und des aktuellen Zustands des Rovers Entscheidungen zu
  treffen.
\item
  \textbf{Benutzerinteraktion:} Obwohl die Hauptinteraktion autonom
  stattfindet, ermöglicht die serielle Ausgabe von Informationen eine
  Überwachung des Systemzustands und der Entscheidungen des Rovers in
  Echtzeit.
\end{itemize}

\newpage

\begin{lstlisting}[language={C++}]
/**
 * @file main.cpp
 * @brief Steuerung eines autonomen Rovers
 * 
 * Zusammenfassend ermöglicht dieses Programm einem autonomen Rover, durch eine Umgebung zu navigieren, indem es kontinuierlich Sensordaten liest, auf diese Daten reagiert und seine Bewegungen entsprechend anpasst, um Hindernissen auszuweichen und bestimmte Ziele zu erreichen.
 */
#include <SoftPWM.h>

// Definiere den Pin für das Ultraschallmodul
#define ULTRASCHALLSENSOR_PIN 10

// Definiere die Pins für die IR-Module
#define IR_RIGHT 7
#define IR_LEFT 8

// Definition der Pins für die Motoren
#define LEFT_MOTOR_FORWARD_PIN 2
#define LEFT_MOTOR_REVERSE_PIN 3
#define RIGHT_MOTOR_FORWARD_PIN 5
#define RIGHT_MOTOR_REVERSE_PIN 4

#define MAX_DISTANCE 30
#define MIN_SAFE_DISTANCE 15
#define MIN_DISTANCE 2
#define FORWARD_SPEED 200
#define CAUTIOUS_SPEED 150
#define BACKWARD_SPEED 200

unsigned long lastActionTime = 0;
const unsigned long backwardDelay = 500;
const unsigned long turnDelay = 1000;
unsigned long lastSensorReadTime = 0;
const long sensorReadInterval = 4; // 4 Millisekunden zwischen den Lesevorgängen

// Globale Variablen für die aktuelle Geschwindigkeit der Motoren
int currentSpeedLeft = 0;
int currentSpeedRight = 0;

// Zustandsvariable als globale Variable deklarieren
enum State {FORWARD, CHECK_OBSTACLE, BACKWARD, TURN_LEFT, REVERSE, AVOID, WAIT, ERROR} state = FORWARD;

void handleNonForwardMovement(unsigned long currentMillis);
void checkObstacle(unsigned long currentMillis);


void setup() {
  Serial.begin(115200);
  SoftPWMBegin();

  pinMode(IR_RIGHT, INPUT);
  pinMode(IR_LEFT, INPUT);
}

void loop() {
  static unsigned long lastSensorReadTime = 0;
  static float distance = -1.0; // Initialwert, um zu erkennen, wann keine Messung vorliegt

  if (currentMillis - lastSensorReadTime >= sensorReadInterval) {
    lastSensorReadTime = currentMillis;
    distance = readSensorData(); // Aktualisiere die Distanz nur, wenn eine neue Messung erfolgt
  }

  if (distance == -1.0) {
      handleErrorState();
      return;
  }

  updateState();
}

void handleErrorState() {
  stopMove(); // Rover anhalten
  Serial.println("Fehlerbehandlung aktiv: Rover gestoppt.");
  // Zusätzliche Fehlerbehandlungsaktionen hier, z.B. Signallicht einschalten, etc.
}

void updateState() {
  unsigned long currentMillis = millis();

  switch (state) {
    case FORWARD:
      if (shouldMoveBackward()) {
        moveBackward(200);
        lastActionTime = currentMillis;
        state = BACKWARD;
      } else if (shouldMoveForward()) {
        moveForward(200);
      }
      break;

    case BACKWARD:
      if (currentMillis - lastActionTime > backwardDelay) {
        state = TURN_LEFT;
        lastActionTime = currentMillis;
      }
      break;

    case TURN_LEFT:
      if (currentMillis - lastActionTime > turnDelay) {
        state = FORWARD;
      }
      break;

    // Weitere Zustände und Logik...
  }
}


float readSensorData() {
  float distances[maxAttempts];
  int validMeasurements = 0;

  for (int attempt = 0; attempt < maxAttempts; attempt++) {
    float distance = performSingleMeasurement();
    if (distance >= minDistance && distance <= maxDistance) {
      distances[validMeasurements++] = distance;
    }
  }

  if (validMeasurements > 0) {
    return calculateMedian(distances, validMeasurements); // Berechne den Median der gültigen Messungen
  } else {
    Serial.println("Fehler: Keine gültige Distanzmessung.");
    return -1.0;
  }
}

float performSingleMeasurement() {
    // Sensor-Auslösungssequenz
    pinMode(ULTRASCHALLSENSOR_PIN, OUTPUT);
    digitalWrite(ULTRASCHALLSENSOR_PIN, LOW);
    delayMicroseconds(2);
    digitalWrite(ULTRASCHALLSENSOR_PIN, HIGH);
    delayMicroseconds(10);
    digitalWrite(ULTRASCHALLSENSOR_PIN, LOW);
    
    pinMode(ULTRASCHALLSENSOR_PIN, INPUT);
    long duration = pulseIn(ULTRASCHALLSENSOR_PIN, HIGH);
    
    // Berechne die Distanz (in cm) basierend auf der Zeitdauer des Echo-Signals
    float distance = (duration / 2.0) * 0.0343;
    
    return distance;
}

int compare(const void* a, const void* b) {
    float fa = *(const float*)a;
    float fb = *(const float*)b;
    return (fa > fb) - (fa < fb);
}

float calculateMedian(float* array, int length) {
    qsort(array, length, sizeof(float), compare); // Sortiere das Array
    
    if (length % 2 == 0) {
        // Wenn die Länge gerade ist, berechne den Durchschnitt der zwei mittleren Werte
        return (array[length / 2 - 1] + array[length / 2]) / 2.0;
    } else {
        // Wenn die Länge ungerade ist, nimm den mittleren Wert
        return array[length / 2];
    }
}



void handleForwardMovement() {
  static unsigned long lastActionTime = 0;
  unsigned long currentMillis = millis();
  static enum { MOVE_FORWARD, MOVE_BACKWARD, TURN_LEFT, WAIT } moveState = MOVE_FORWARD;

  float distance = readSensorData();

  switch (moveState) {
    case MOVE_FORWARD:
      if (distance > 30) {
        moveForward(200);
      } else if (distance < 15 && distance > 2) {
        moveBackward(200);
        lastActionTime = currentMillis;
        moveState = MOVE_BACKWARD;
      } else {
        moveForward(150);
      }
      break;
    case MOVE_BACKWARD:
      if (currentMillis - lastActionTime > 500) {
        backLeft(150);
        lastActionTime = currentMillis;
        moveState = TURN_LEFT;
      }
      break;
    case TURN_LEFT:
      if (currentMillis - lastActionTime > 1000) {
        moveState = MOVE_FORWARD; // Rückkehr zum Vorwärtsbewegen oder zu einem anderen Zustand
      }
      break;
  }
}


void handleNonForwardMovement(unsigned long currentMillis) {
  if (currentMillis - lastActionTime >= backwardDelay && state == BACKWARD) {
    backLeft(150);
    lastActionTime = currentMillis;
    state = TURN_LEFT;
  } else if (currentMillis - lastActionTime >= turnDelay && state == TURN_LEFT) {
    state = CHECK_OBSTACLE;
  }
}

void checkObstacle(unsigned long currentMillis) {
  float distance = readSensorData();
  if (distance > MAX_DISTANCE) {
    state = FORWARD;
  } else {
    moveBackward(BACKWARD_SPEED);
    lastActionTime = currentMillis;
    state = BACKWARD;
  }
}


// Funktion zur schrittweisen Anpassung der Geschwindigkeit
void adjustSpeed(int* currentSpeed, int targetSpeed, int adjustmentStep) {
  if (*currentSpeed < targetSpeed) {
    *currentSpeed += adjustmentStep;
    if (*currentSpeed > targetSpeed) { // Verhindert Übersteuern
      *currentSpeed = targetSpeed;
    }
  } else if (*currentSpeed > targetSpeed) {
    *currentSpeed -= adjustmentStep;
    if (*currentSpeed < targetSpeed) { // Verhindert Übersteuern
      *currentSpeed = targetSpeed;
    }
  }
}

// Beispiel für die Anwendung in moveForward
void moveForward(int targetSpeed) {
  adjustSpeed(&currentSpeedLeft, targetSpeed, 10); // Annahme: 10 als Anpassungsschritt
  adjustSpeed(&currentSpeedRight, targetSpeed, 10);
  
  SoftPWMSet(LEFT_MOTOR_FORWARD_PIN, currentSpeedLeft);
  SoftPWMSet(RIGHT_MOTOR_FORWARD_PIN, currentSpeedRight);
}

void moveBackward(int targetSpeed) {
  adjustSpeed(&currentSpeedLeft, -targetSpeed, 10); // Negative Geschwindigkeit für Rückwärts
  adjustSpeed(&currentSpeedRight, -targetSpeed, 10);
  
  SoftPWMSet(LEFT_MOTOR_REVERSE_PIN, abs(currentSpeedLeft));
  SoftPWMSet(RIGHT_MOTOR_REVERSE_PIN, abs(currentSpeedRight));
}


void backLeft(int targetSpeed) {
  adjustSpeed(&currentSpeedLeft, 0, 10); // Reduziere die Geschwindigkeit des linken Motors auf 0
  adjustSpeed(&currentSpeedRight, targetSpeed, 10); // Erhöhe die Geschwindigkeit des rechten Motors
  
  SoftPWMSet(LEFT_MOTOR_FORWARD_PIN, 0); // Stelle sicher, dass der linke Motor nicht vorwärts läuft
  SoftPWMSet(LEFT_MOTOR_REVERSE_PIN, 0); // Der linke Motor bleibt in diesem Fall gestoppt
  SoftPWMSet(RIGHT_MOTOR_FORWARD_PIN, currentSpeedRight); // Setze die neue Geschwindigkeit für den rechten Motor
  SoftPWMSet(RIGHT_MOTOR_REVERSE_PIN, 0); // Stelle sicher, dass der rechte Motor nicht rückwärts läuft
}

void backRight(int targetSpeed) {
  adjustSpeed(&currentSpeedLeft, targetSpeed, 10); // Erhöhe die Geschwindigkeit des linken Motors
  adjustSpeed(&currentSpeedRight, 0, 10); // Reduziere die Geschwindigkeit des rechten Motors auf 0
  
  SoftPWMSet(LEFT_MOTOR_FORWARD_PIN, currentSpeedLeft); // Setze die neue Geschwindigkeit für den linken Motor
  SoftPWMSet(LEFT_MOTOR_REVERSE_PIN, 0); // Stelle sicher, dass der linke Motor nicht rückwärts läuft
  SoftPWMSet(RIGHT_MOTOR_FORWARD_PIN, 0); // Der rechte Motor bleibt in diesem Fall gestoppt
  SoftPWMSet(RIGHT_MOTOR_REVERSE_PIN, 0); // Stelle sicher, dass der rechte Motor nicht rückwärts läuft
}


void turnLeft(int targetSpeed) {
  adjustSpeed(&currentSpeedLeft, -targetSpeed, 10); // Linken Motor rückwärts für Linksdrehung
  adjustSpeed(&currentSpeedRight, targetSpeed, 10); // Rechten Motor vorwärts
  
  SoftPWMSet(LEFT_MOTOR_REVERSE_PIN, abs(currentSpeedLeft));
  SoftPWMSet(RIGHT_MOTOR_FORWARD_PIN, currentSpeedRight);
}


void turnRight(int targetSpeed) {
  adjustSpeed(&currentSpeedLeft, targetSpeed, 10); // Linken Motor vorwärts für Rechtsdrehung
  adjustSpeed(&currentSpeedRight, -targetSpeed, 10); // Rechten Motor rückwärts
  
  SoftPWMSet(LEFT_MOTOR_FORWARD_PIN, currentSpeedLeft);
  SoftPWMSet(RIGHT_MOTOR_REVERSE_PIN, abs(currentSpeedRight));
}


void stopMove() {
  // Stoppe alle Motoren
  SoftPWMSet(LEFT_MOTOR_FORWARD_PIN, 0);
  SoftPWMSet(LEFT_MOTOR_REVERSE_PIN, 0);
  SoftPWMSet(RIGHT_MOTOR_FORWARD_PIN, 0);
  SoftPWMSet(RIGHT_MOTOR_REVERSE_PIN, 0);
}
\end{lstlisting}

\hypertarget{uxfcberpruxfcfung}{%
\subsubsection{Überprüfung}\label{ueberpruefung}}

\begin{itemize}
\item
  \textbf{Modulare Struktur:} Der Code ist gut strukturiert und modular
  aufgebaut, was die Wartbarkeit und Erweiterbarkeit erleichtert. Die
  Verwendung von Funktionen für spezifische Aufgaben wie
  \verb|moveForward|,
  \verb|moveBackward|,
  \verb|backLeft|,
  \verb|backRight|,
  \verb|turnLeft|,
  \verb|turnRight|, und
  \verb|stopMove| verbessert die Lesbarkeit und
  Wiederverwendbarkeit des Codes.
\item
  \textbf{Zustandsverwaltung:} Der Einsatz eines Enum-basierten
  Zustandsautomaten (\verb|State|) für die Verwaltung
  der Bewegungslogik ist eine effektive Methode, um den aktuellen
  Zustand des Rovers zu verfolgen und entsprechend auf Sensordaten zu
  reagieren.
\item
  \textbf{Sensorabfrage und Datenverarbeitung:} Die Funktion
  \verb|readSensorData| implementiert eine robuste
  Logik für die Ultraschallmessung, einschließlich der Berechnung des
  Medians aus mehreren Messungen, was die Genauigkeit der Distanzmessung
  verbessert und die Anfälligkeit für Ausreißer reduziert.
\end{itemize}

\hypertarget{verbesserungsvorschluxe4ge}{%
\subsubsection{Verbesserungsvorschläge}\label{verbesserungsvorschlaege}}

\begin{enumerate}
\def\labelenumi{\arabic{enumi}.}
\item
  \textbf{Optimierung der Sensorleseintervalle:} Derzeit wird ein festes
  Intervall von 4 Millisekunden zwischen den Lesevorgängen verwendet. Je
  nach Anwendungsfall und erforderlicher Reaktionsgeschwindigkeit könnte
  es vorteilhaft sein, dieses Intervall dynamisch anzupassen, um die
  Effizienz zu optimieren und Energie zu sparen.
\item
  \textbf{Geschwindigkeitsanpassung:} Die Funktionen zur Anpassung der
  Geschwindigkeit (\verb|adjustSpeed|) und zur
  Bewegungskontrolle bieten eine Grundlage für präzise Steuerung. Für
  eine noch feinere Kontrolle könnten zusätzliche Faktoren wie
  Beschleunigung und Verzögerung eingeführt werden, um ruckartige
  Bewegungen zu minimieren und die mechanische Belastung des Rovers zu
  reduzieren.
\item
  \textbf{Erweiterte Fehlerbehandlung:} Obwohl eine grundlegende
  Fehlerbehandlung implementiert ist, könnte die Robustheit des Systems
  durch detailliertere Reaktionen auf spezifische Fehlerzustände oder
  durch Implementierung eines Wiederholungsmechanismus für
  fehlgeschlagene Sensorlesevorgänge weiter verbessert werden.
\item
  \textbf{Integration zusätzlicher Sensortypen:} Die Erweiterung des
  Systems um zusätzliche Sensortypen, wie Infrarot- oder Lidar-Sensoren,
  könnte die Hinderniserkennung und die Navigationsfähigkeiten des
  Rovers verbessern, insbesondere in komplexen oder variablen
  Umgebungen.
\item
  \textbf{Benutzerinteraktion und Feedback:} Die Implementierung von
  Feedback-Mechanismen, beispielsweise durch LED-Signale oder Töne,
  könnte die Benutzererfahrung verbessern und nützliche Informationen
  über den Betriebszustand des Rovers liefern.
\end{enumerate}

Zusammenfassend ist der bereitgestellte Code ein solider Ausgangspunkt
für die Entwicklung eines autonomen Rovers mit fortschrittlichen
Navigations- und Hindernisvermeidungsfähigkeiten. Durch die
Berücksichtigung der genannten Verbesserungsvorschläge könnte die
Leistungsfähigkeit und Zuverlässigkeit des Systems weiter gesteigert
werden.

\hypertarget{intelligentes-folgesystem}{%
\subsection{Intelligentes Folgesystem}\label{intelligentes-folgesystem}}

\begin{itemize}

\item
  Modifikation des Codes, damit der Rover sich auf bewegende Objekte
  zubewegt, basierend auf der Distanzmessung durch den Ultraschallsensor
  und Objekterkennung durch Infrarotsensoren.
\end{itemize}

\newpage

\begin{lstlisting}
\end{lstlisting}

\newpage

\begin{lstlisting}[language={C++}]
/**
 * @file main.cpp
 * @brief Intelligentes Folgesystem
 * 
 * Modifikation des Codes, damit der Rover sich auf bewegende Objekte zubewegt,
 * basierend auf der Distanzmessung durch den Ultraschallsensor und 
 * Objekterkennung durch Infrarotsensoren.
 */
#include <SoftPWM.h>

// Define the pin for the ultrasonic module
#define ULTRASCHALLSENSOR_PIN 10

// Define the pins for the IR modules
#define IR_RIGHT 7
#define IR_LEFT 8

// Definition der Pins für die linken Motoren A, B, C
#define LEFT_MOTOR_FORWARD_PIN 2 // Pin für Vorwärtsbewegung der linken Motoren (A, B, C)
#define LEFT_MOTOR_REVERSE_PIN 3 // Pin für Rückwärtsbewegung der linken Motoren (A, B, C)

// Definition der Pins für die rechten Motoren D, E, F
#define RIGHT_MOTOR_FORWARD_PIN 5 // Pin für Vorwärtsbewegung der rechten Motoren (D, E, F)
#define RIGHT_MOTOR_REVERSE_PIN 4 // Pin für Rückwärtsbewegung der rechten Motoren (D, E, F)
void setup() {

  // Initialize serial communication for debugging
  Serial.begin(115200);

  // Initialize SoftPWM
  SoftPWMBegin();

  // Set the IR module pins as inputs
  pinMode(IR_RIGHT, INPUT);
  pinMode(IR_LEFT, INPUT);
}

void loop() {

  float distance = readSensorData();

  // Read values from IR sensors
  int rightValue = digitalRead(IR_RIGHT);
  int leftValue = digitalRead(IR_LEFT);

  if (distance > 5 && distance < 30) {
    moveForward(150);
  }
  // Based on IR sensor readings, control rover's movements
  else if (rightValue == 0 && leftValue == 1) {  // Right ir module blocked
    turnRight(150);
  } 
  else if (rightValue == 1 && leftValue == 0) {  // Left ir module blocked
    turnLeft(150);
  } 
  else {  // Paths clear
    stopMove();
  }
}


float readSensorData() {
  // A 4ms delay is required, otherwise the reading may be 0
  delay(4);

  //Set to OUTPUT to send signal
  pinMode(ULTRASCHALLSENSOR_PIN, OUTPUT);

  // Clear the trigger pin
  digitalWrite(ULTRASCHALLSENSOR_PIN, LOW);
  delayMicroseconds(2);

  // Trigger the sensor by sending a high pulse for 10us
  digitalWrite(ULTRASCHALLSENSOR_PIN, HIGH);
  delayMicroseconds(10);

  // Set the trigger pin back to low
  digitalWrite(ULTRASCHALLSENSOR_PIN, LOW);

  //Set to INPUT to read
  pinMode(ULTRASCHALLSENSOR_PIN, INPUT);

  // pulseIn returns the duration of the pulse on the pin
  float duration = pulseIn(ULTRASCHALLSENSOR_PIN, HIGH);

  // Calculate the distance (in cm) based on the speed of sound (340 m/s or 0.034 cm/us)
  float distance = duration * 0.034 / 2;

  return distance;
}

void moveForward(int speed) {
  // Set the left motors rotate counterclockwise
  SoftPWMSet(LEFT_MOTOR_FORWARD_PIN, speed);
  SoftPWMSet(LEFT_MOTOR_REVERSE_PIN, 0);

  // Set the right motors rotate clockwise
  SoftPWMSet(RIGHT_MOTOR_FORWARD_PIN, 0);
  SoftPWMSet(RIGHT_MOTOR_REVERSE_PIN, speed);
}

void moveBackward(int speed) {
  // Set the left motors rotate clockwise
  SoftPWMSet(LEFT_MOTOR_FORWARD_PIN, 0);
  SoftPWMSet(LEFT_MOTOR_REVERSE_PIN, speed);

  // Set the right motors rotate counterclockwise
  SoftPWMSet(RIGHT_MOTOR_FORWARD_PIN, speed);
  SoftPWMSet(RIGHT_MOTOR_REVERSE_PIN, 0);
}

void turnLeft(int speed) {
  // Set al motors to rotate clockwise
  SoftPWMSet(LEFT_MOTOR_FORWARD_PIN, 0);
  SoftPWMSet(LEFT_MOTOR_REVERSE_PIN, speed);
  SoftPWMSet(RIGHT_MOTOR_FORWARD_PIN, 0);
  SoftPWMSet(RIGHT_MOTOR_REVERSE_PIN, speed);
}

void turnRight(int speed) {
  // Set all motors to rotate counterclockwise
  SoftPWMSet(LEFT_MOTOR_FORWARD_PIN, speed);
  SoftPWMSet(LEFT_MOTOR_REVERSE_PIN, 0);
  SoftPWMSet(RIGHT_MOTOR_FORWARD_PIN, speed);
  SoftPWMSet(RIGHT_MOTOR_REVERSE_PIN, 0);
}

void stopMove() {
  // Stop all the motors
  SoftPWMSet(LEFT_MOTOR_FORWARD_PIN, 0);
  SoftPWMSet(LEFT_MOTOR_REVERSE_PIN, 0);
  SoftPWMSet(RIGHT_MOTOR_FORWARD_PIN, 0);
  SoftPWMSet(RIGHT_MOTOR_REVERSE_PIN, 0);
}
\end{lstlisting}

\hypertarget{reflexion-und-lernprozess}{%
\subsection{Reflexion und Lernprozess}\label{reflexion-und-lernprozess}}

\begin{itemize}
\item
  Warum denken Sie, haben wir im Hindernisvermeidungssystem das
  Hindernisvermeidungsmodul vor dem Ultraschallsensor priorisiert und
  umgekehrt im Folgesystem?
\item
  Wie würde sich das Ergebnis ändern, wenn wir die Reihenfolge, in der
  diese Module im Code überprüft werden, tauschen würden?
\end{itemize}

\hypertarget{energieeffizienz}{%
\subsection{Energieeffizienz}\label{energieeffizienz}}

\hypertarget{optimierung-der-bewegungsgeschwindigkeit}{%
\subsubsection{\texorpdfstring{1. \textbf{Optimierung der
Bewegungsgeschwindigkeit}}{1. Optimierung der Bewegungsgeschwindigkeit}}\label{optimierung-der-bewegungsgeschwindigkeit}}

Eine der effektivsten Methoden zur Reduzierung des Energieverbrauchs ist
die Optimierung der Reisegeschwindigkeit. Studien haben gezeigt, dass es
oft eine optimale Geschwindigkeit gibt, bei der der Energieverbrauch
minimiert wird. Das Finden dieser optimalen Geschwindigkeit erfordert
Tests und Anpassungen basierend auf den spezifischen Eigenschaften des
Rovers und seiner Umgebung.

\hypertarget{effiziente-beschleunigungs--und-verzuxf6gerungsprofile}{%
\subsubsection{\texorpdfstring{2. \textbf{Effiziente Beschleunigungs-
und
Verzögerungsprofile}}{2. Effiziente Beschleunigungs- und Verzögerungsprofile}}\label{effiziente-beschleunigungs--und-verzoegerungsprofile}}

Statt einer linearen Beschleunigung oder Verzögerung können effizientere
Profile, wie z.B. eine s-förmige (sigmoidale) oder stufenweise
Beschleunigung, den Energieverbrauch reduzieren. Diese Profile
ermöglichen einen sanfteren Übergang von Stillstand zu Bewegung und
umgekehrt, was die mechanische Belastung und den Energieverbrauch
verringern kann.

\hypertarget{energie-ruxfcckgewinnungssysteme}{%
\subsubsection{\texorpdfstring{3.
\textbf{Energie-Rückgewinnungssysteme}}{3. Energie-Rückgewinnungssysteme}}\label{energie-rueckgewinnungssysteme}}

In einigen fortschrittlichen Systemen kann die Implementierung von
Energie-Rückgewinnungstechnologien während der Verzögerung oder beim
Bergabfahren den Energieverbrauch senken. Obwohl die Implementierung
solcher Systeme in kleineren oder einfacheren Robotern möglicherweise
nicht praktikabel ist, stellt sie in größeren Systemen eine wertvolle
Möglichkeit zur Effizienzsteigerung dar.

\hypertarget{anpassung-an-die-umgebung}{%
\subsubsection{\texorpdfstring{4. \textbf{Anpassung an die
Umgebung}}{4. Anpassung an die Umgebung}}\label{anpassung-an-die-umgebung}}

Die Anpassung der Bewegungsstrategien an die spezifischen Bedingungen
der Umgebung kann ebenfalls zur Energieeinsparung beitragen. Zum
Beispiel kann das Vermeiden von unnötigen Stopps und Starts in hügeligem
Gelände oder das Anpassen der Geschwindigkeit basierend auf dem
Untergrund (Asphalt vs.~Sand) den Gesamtenergieverbrauch reduzieren.

\hypertarget{einsatz-von-energiesparmodi}{%
\subsubsection{\texorpdfstring{5. \textbf{Einsatz von
Energiesparmodi}}{5. Einsatz von Energiesparmodi}}\label{einsatz-von-energiesparmodi}}

Für Zeiten, in denen der Rover nicht aktiv eine Aufgabe ausführt, können
Energiesparmodi implementiert werden, die die Energieverbrauch von
Sensoren, Kommunikationssystemen und anderen nicht kritischen
Komponenten reduzieren.

\hypertarget{softwareseitige-optimierungen}{%
\subsubsection{\texorpdfstring{6. \textbf{Softwareseitige
Optimierungen}}{6. Softwareseitige Optimierungen}}\label{softwareseitige-optimierungen}}

Die Effizienz der Software, einschließlich der Algorithmen für die
Wegfindung und Hindernisvermeidung, spielt ebenfalls eine Rolle bei der
Energieeffizienz. Effizientere Algorithmen können dazu beitragen, den
Energieverbrauch zu senken, indem sie den Rover auf kürzeren oder
weniger energieintensiven Pfaden navigieren.

\hypertarget{fazit}{%
\subsubsection{Fazit}\label{fazit}}

Die Berücksichtigung der Energieeffizienz in der
Bewegungssteuerungsstrategie eines Rovers ist entscheidend für die
Maximierung der Batterielebensdauer und die Minimierung des
Gesamtenergieverbrauchs. Durch die Implementierung dieser Strategien
kann ein Rover länger und effizienter in seiner Umgebung operieren, was
besonders wichtig in Anwendungen ist, wo die Energieversorgung begrenzt
oder der Zugang zur Nachladung eingeschränkt ist. % Platzhalter

%% Optional Anhang
%\clearpage
%\appendix

\clearpage
\printbibliography
\end{document}