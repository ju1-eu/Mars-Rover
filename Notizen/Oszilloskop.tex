% ju 19-Feb-24 Oszilloskop.tex
\documentclass{vorlage-design-main}
\usepackage[utf8]{inputenc}
\usepackage{longtable}
\usepackage{blindtext,alltt}
%% Ganze Überschrift
\title{Thema}

%% Kürzerer Titel zur Verwendung im Seitenkopf
\runningtitle{Kurztitel}
\author{Jan Unger}
% \author{2.}
\date{\today}

%% Die .bib-Datei mit vollständigen Referenzen zur Verwendung mit biblatex. articleclass lädt das Paket biblatex-chicago mit Anpassungen
\addbibresource{literatur.bib}

\begin{document}

\maketitle

\begin{abstract}

\end{abstract}

\hypertarget{empfehlung-eines-oszilloskops}{%
\subsection{Empfehlung eines
Oszilloskops}\label{empfehlung-eines-oszilloskops}}

\hypertarget{faktoren-zur-beruxfccksichtigung}{%
\subsubsection{Faktoren zur
Berücksichtigung}\label{faktoren-zur-beruecksichtigung}}

\begin{itemize}

\item
  \textbf{Bandbreite}: Die Bandbreite des Oszilloskops sollte mindestens
  so hoch sein wie die höchste Frequenz, die Sie messen möchten.
\item
  \textbf{Abtastrate}: Eine höhere Abtastrate ermöglicht eine genauere
  Darstellung schneller Signale.
\item
  \textbf{Kanäle}: Überlegen Sie, wie viele Signale Sie gleichzeitig
  messen müssen. Zwei Kanäle sind üblich, aber für komplexere Analysen
  könnten vier oder mehr Kanäle nötig sein.
\item
  \textbf{Speichertiefe}: Eine größere Speichertiefe ermöglicht es
  Ihnen, längere Zeiträume des Signals bei hoher Abtastrate zu
  speichern.
\item
  \textbf{Benutzerfreundlichkeit}: Die Bedienoberfläche sollte intuitiv
  sein, und das Gerät sollte über ausreichende Mess- und
  Analysefunktionen verfügen.
\item
  \textbf{Preis}: Oszilloskope reichen von günstigen Einsteigermodellen
  bis hin zu hochwertigen Geräten für professionelle Anwendungen.
\end{itemize}

\hypertarget{empfehlungen}{%
\subsubsection{Empfehlungen}\label{empfehlungen}}

\begin{enumerate}
\def\labelenumi{\arabic{enumi}.}

\item
  \textbf{Einsteiger}: Rigol DS1054Z

  \begin{itemize}
  
  \item
    Eine beliebte Wahl für Hobbyisten und Bildungszwecke. Es bietet eine
    gute Bandbreite (50 MHz, software-upgradebar) und vier Kanäle zu
    einem erschwinglichen Preis.
  \end{itemize}
\item
  \textbf{Mittleres Segment}: Siglent SDS1104X-E

  \begin{itemize}
  
  \item
    Ein Schritt nach oben mit einer Bandbreite von 100 MHz und vier
    Kanälen. Es bietet eine hohe Abtastrate und eine benutzerfreundliche
    Oberfläche, was es für ernsthafte Hobbyisten und semiprofessionelle
    Anwendungen geeignet macht.
  \end{itemize}
\item
  \textbf{Hochwertig/Professionell}: Keysight Technologies InfiniiVision
  3000T X-Series

  \begin{itemize}
  
  \item
    Für professionelle Anwender, die eine höhere Bandbreite (bis zu
    mehreren GHz), ausgezeichnete Abtastraten, tiefe Speichertiefe und
    fortschrittliche Analysefunktionen benötigen. Diese Oszilloskope
    sind teurer, bieten aber hervorragende Leistung und Zuverlässigkeit.
  \end{itemize}
\item
  \textbf{Portabel}: Fluke 123B Industrial ScopeMeter

  \begin{itemize}
  
  \item
    Wenn Mobilität wichtig ist, sind die ScopeMeter von Fluke robust und
    tragbar, ideal für Feldarbeiten oder wenn Sie ein Oszilloskop an
    verschiedenen Orten einsetzen müssen.
  \end{itemize}
\end{enumerate}

\hypertarget{siglent-sds1104x-e-oszilloskop}{%
\subsection{Siglent SDS1104X-E
Oszilloskop}\label{siglent-sds1104x-e-oszilloskop}}

Mit dem Siglent SDS1104X-E Oszilloskop können Sie eine Vielzahl von
Messungen durchführen, die für die Entwicklung und Fehlersuche an einem
autonomen Rover wie dem in Ihrem Code beschriebenen sehr nützlich sind.

\hypertarget{ultraschallsensor-und-ir-sensoren}{%
\subsubsection{Ultraschallsensor und
IR-Sensoren}\label{ultraschallsensor-und-ir-sensoren}}

\begin{itemize}

\item
  \textbf{Signalform}: Überprüfen Sie die Form der Trigger-Signale und
  der Echo-Signale des Ultraschallsensors sowie der Signale der
  IR-Sensoren, um sicherzustellen, dass sie korrekt generiert und
  empfangen werden.
\item
  \textbf{Timing}: Messen Sie die Zeitdauer zwischen dem Senden und
  Empfangen der Signale, um die Distanzmessungen zu verifizieren und zu
  kalibrieren.
\item
  \textbf{Störungen}: Suchen Sie nach möglichen elektromagnetischen
  Störungen oder Signalübersprechen zwischen den Sensoren, die die
  Messgenauigkeit beeinträchtigen könnten.
\end{itemize}

\hypertarget{x-getriebemotoren}{%
\subsubsection{6x Getriebemotoren}\label{x-getriebemotoren}}

\begin{itemize}

\item
  \textbf{PWM-Signal}: Analysieren Sie die PWM-Signale, die an die
  Motoren gesendet werden, um sicherzustellen, dass die Geschwindigkeit
  und Richtung der Motoren wie erwartet gesteuert werden.
\item
  \textbf{Spannungsschwankungen}: Beobachten Sie die Spannung an den
  Motoren während des Betriebs, um festzustellen, ob es zu unerwünschten
  Spannungseinbrüchen kommt, die die Motorleistung beeinträchtigen
  könnten.
\item
  \textbf{Motorstart- und Stoppzeiten}: Messen Sie, wie schnell die
  Motoren auf Befehle reagieren, um die Effizienz der Bewegungssteuerung
  zu optimieren.
\end{itemize}

\hypertarget{batterie}{%
\subsubsection{18650 Batterie}\label{batterie}}

\begin{itemize}

\item
  \textbf{Spannungsverlauf}: Überwachen Sie die Spannung der Batterie
  unter Last, um die Batterielebensdauer und den Einfluss auf die
  Systemleistung zu bewerten.
\item
  \textbf{Stromaufnahme}: Messen Sie den Stromverbrauch des gesamten
  Systems, um die Energieeffizienz zu beurteilen und sicherzustellen,
  dass die Batterie die erforderliche Leistung liefern kann.
\end{itemize}

\hypertarget{messungen}{%
\subsection{Messungen}\label{messungen}}

\hypertarget{ultraschallsensor-und-ir-sensoren-messen}{%
\subsubsection{Ultraschallsensor und IR-Sensoren
messen}\label{ultraschallsensor-und-ir-sensoren-messen}}

\hypertarget{signalform}{%
\paragraph{Signalform}\label{signalform}}

\begin{itemize}

\item
  \textbf{Setup}: Verbinden Sie den Ausgangspunkt des Ultraschallsensors
  (TRIG) und den Eingangspunkt (ECHO) sowie die Ausgänge der IR-Sensoren
  mit den verschiedenen Kanälen Ihres Oszilloskops. Verwenden Sie
  Tastköpfe mit passender Bandbreite und stellen Sie sicher, dass sie
  korrekt auf das Signalniveau eingestellt sind (x1 oder x10, abhängig
  von der erwarteten Signalstärke).
\item
  \textbf{Messung}: Aktivieren Sie den Sensor und beobachten Sie die
  Trigger- und Echo-Signale auf dem Bildschirm. Achten Sie auf die Form
  des Signals, um sicherzustellen, dass keine Verzerrungen oder
  unerwartete Änderungen vorliegen.
\end{itemize}

\hypertarget{timing}{%
\paragraph{Timing}\label{timing}}

\begin{itemize}

\item
  \textbf{Messung}: Verwenden Sie die Zeitbasis und die Cursor-Funktion
  Ihres Oszilloskops, um die Zeit zwischen dem Senden des
  Trigger-Signals und dem Empfang des Echo-Signals zu messen. Diese Zeit
  wird verwendet, um die Entfernung zum Objekt zu berechnen. Vergleichen
  Sie diese Zeit mit den erwarteten Werten, um die Genauigkeit Ihrer
  Messung zu verifizieren.
\end{itemize}

\hypertarget{stuxf6rungen}{%
\paragraph{Störungen}\label{stoerungen}}

\begin{itemize}

\item
  \textbf{Untersuchung}: Überprüfen Sie das Signal auf unerwünschte
  Spitzen oder Rauschen, die auf elektromagnetische Störungen hindeuten
  könnten. Platzieren Sie die Sensoren möglicherweise näher beieinander
  oder fügen Sie Abschirmungen hinzu, um das Signalübersprechen zu
  minimieren.
\end{itemize}

\hypertarget{x-getriebemotoren-messen}{%
\subsubsection{6x Getriebemotoren
messen}\label{x-getriebemotoren-messen}}

\hypertarget{pwm-signal}{%
\paragraph{PWM-Signal}\label{pwm-signal}}

\begin{itemize}

\item
  \textbf{Messung}: Verbinden Sie das Oszilloskop mit den PWM-Ausgängen,
  die die Motoren steuern. Beobachten Sie die Pulsbreite und die
  Frequenz der PWM-Signale, um sicherzustellen, dass sie den erwarteten
  Werten entsprechen, die für die Steuerung der Motorgeschwindigkeit und
  -richtung benötigt werden.
\end{itemize}

\hypertarget{spannungsschwankungen}{%
\paragraph{Spannungsschwankungen}\label{spannungsschwankungen}}

\begin{itemize}

\item
  \textbf{Überwachung}: Beobachten Sie die Spannung an den Motoren
  während des Betriebs. Verwenden Sie die AC-Kopplung Ihres
  Oszilloskops, um die Spannungsschwankungen zu sehen, oder die
  DC-Kopplung, um die Gesamtspannung einschließlich der Schwankungen zu
  messen.
\end{itemize}

\hypertarget{motorstart--und-stoppzeiten}{%
\paragraph{Motorstart- und
Stoppzeiten}\label{motorstart--und-stoppzeiten}}

\begin{itemize}

\item
  \textbf{Analyse}: Messen Sie, wie schnell die Motoren auf
  Steuerbefehle reagieren, indem Sie die Zeit zwischen dem Ändern des
  PWM-Signals und der Reaktion des Motors messen. Dies kann Ihnen
  helfen, die Effizienz der Bewegungssteuerung zu verbessern.
\end{itemize}

\hypertarget{batterie-messen}{%
\subsubsection{18650 Batterie messen}\label{batterie-messen}}

\hypertarget{spannungsverlauf}{%
\paragraph{Spannungsverlauf}\label{spannungsverlauf}}

\begin{itemize}

\item
  \textbf{Überwachung}: Schließen Sie das Oszilloskop an die Batterie
  an, um die Spannung unter Last zu messen. Beobachten Sie, wie die
  Spannung variiert, wenn der Rover verschiedene Aktionen durchführt, um
  sicherzustellen, dass die Batterie stabil bleibt und keine
  unerwünschten Spannungseinbrüche auftreten.
\end{itemize}

\hypertarget{stromaufnahme}{%
\paragraph{Stromaufnahme}\label{stromaufnahme}}

\begin{itemize}

\item
  \textbf{Messung}: Um den Stromverbrauch zu messen, benötigen Sie
  möglicherweise ein zusätzliches Strommessgerät oder einen
  Shunt-Widerstand, über den Sie die Spannung messen können, um den
  durch ihn fließenden Strom zu berechnen. Verbinden Sie das Oszilloskop
  in Serie mit dem Strommessgerät oder dem Shunt, um die Veränderungen
  der Stromaufnahme während des Betriebs zu beobachten.
\end{itemize}

\hypertarget{shunt-widerstands-fuxfcr-die-strommessung}{%
\subsection{Shunt-Widerstands für die
Strommessung}\label{shunt-widerstands-fuer-die-strommessung}}

Die Wahl eines geeigneten Shunt-Widerstands für die Strommessung in
Ihrem Rover-Projekt hängt von mehreren Faktoren ab, einschließlich des
maximalen Stroms, der durch das System fließt, der Spannung, die Ihr
Messgerät sicher handhaben kann, und der Auflösung, die Sie erreichen
möchten. Hier sind einige Überlegungen und Empfehlungen:

\hypertarget{uxfcberlegungen}{%
\subsubsection{Überlegungen}\label{ueberlegungen}}

\begin{enumerate}
\def\labelenumi{\arabic{enumi}.}
\item
  \textbf{Maximaler Strom}: Bestimmen Sie den maximalen Strom, der durch
  Ihren Rover fließen wird. Dieser Wert ist wichtig, um einen
  Shunt-Widerstand mit einer geeigneten Nennleistung zu wählen.
\item
  \textbf{Spannungsabfall}: Der Shunt-Widerstand erzeugt einen
  Spannungsabfall proportional zum durchfließenden Strom. Dieser
  Spannungsabfall sollte groß genug sein, um messbar zu sein, aber nicht
  so groß, dass er die Leistung Ihres Systems beeinträchtigt. Ein
  typischer Wert für den Spannungsabfall über einen Shunt-Widerstand
  liegt zwischen 50 mV und 200 mV bei maximalem Strom.
\item
  \textbf{Auflösung und Genauigkeit}: Die Auflösung Ihres Messgeräts
  (z.B. eines Oszilloskops oder eines Multimeters im Strommessmodus)
  bestimmt, wie klein der Widerstandswert des Shunt sein kann, ohne dass
  die Messgenauigkeit beeinträchtigt wird.
\end{enumerate}

\hypertarget{empfehlungen-shunt-widerstands-fuxfcr-die-strommessung}{%
\subsubsection{Empfehlungen Shunt-Widerstands für die
Strommessung}\label{empfehlungen-shunt-widerstands-fuer-die-strommessung}}

\begin{itemize}
\item
  \textbf{Für niedrige Ströme (bis zu 1 A)}: Ein Widerstandswert von 0,1
  Ohm bis 1 Ohm mit einer Nennleistung von mindestens 1 W könnte
  geeignet sein. Dies würde einen messbaren Spannungsabfall erzeugen,
  ohne zu viel Leistung zu verbrauchen.
\item
  \textbf{Für mittlere Ströme (1 A bis 10 A)}: Ein Widerstandswert im
  Bereich von 10 mOhm bis 100 mOhm mit einer Nennleistung von 2 W bis 5
  W könnte passend sein. Für solche Anwendungen sind
  Präzisions-Shunt-Widerstände mit niedrigem Temperaturkoeffizienten und
  hoher Genauigkeit verfügbar.
\item
  \textbf{Für höhere Ströme (über 10 A)}: Sie benötigen möglicherweise
  einen spezialisierten Shunt-Widerstand mit einem sehr niedrigen
  Widerstandswert (z.B. 1 mOhm bis 10 mOhm) und einer hohen Nennleistung
  (10 W oder mehr), um die Wärmeentwicklung zu bewältigen.
\end{itemize}

\hypertarget{spezifisches-produkt}{%
\subsubsection{Spezifisches Produkt}\label{spezifisches-produkt}}

Ein spezifisches Produkt zu empfehlen, ohne genauere Details zu Ihrem
Projekt zu kennen, ist schwierig. Allerdings sind Shunt-Widerstände von
Herstellern wie Vishay, Bourns oder TE Connectivity allgemein als
hochwertig anerkannt. Ein Beispiel für einen häufig verwendeten
Shunt-Widerstand ist der Vishay Dale WSBS8518\ldots{} Serie für höhere
Stromanwendungen, der eine Nennleistung von bis zu 36 W aufweisen kann
und für Strommessungen in Automotive- und Industrieanwendungen
entwickelt wurde. % Platzhalter

%% Optional Anhang
%\clearpage
%\appendix

\clearpage
\printbibliography
\end{document}