% ju 19-Feb-24 Oszilloskop.tex
\documentclass{vorlage-design-main}
\usepackage[utf8]{inputenc}
\usepackage{longtable}
\usepackage{blindtext,alltt}
%% Ganze Überschrift
\title{Thema}

%% Kürzerer Titel zur Verwendung im Seitenkopf
\runningtitle{Kurztitel}
\author{Jan Unger}
% \author{2.}
\date{\today}

%% Die .bib-Datei mit vollständigen Referenzen zur Verwendung mit biblatex. articleclass lädt das Paket biblatex-chicago mit Anpassungen
\addbibresource{literatur.bib}

\begin{document}

\maketitle

\begin{abstract}

\end{abstract}

\hypertarget{empfehlung-eines-oszilloskops}{%
\subsection{Empfehlung eines
Oszilloskops}\label{empfehlung-eines-oszilloskops}}

\hypertarget{faktoren-zur-beruxfccksichtigung}{%
\subsubsection{Faktoren zur
Berücksichtigung}\label{faktoren-zur-beruecksichtigung}}

\begin{itemize}

\item
  \textbf{Bandbreite}: Die Bandbreite des Oszilloskops sollte mindestens
  so hoch sein wie die höchste Frequenz, die Sie messen möchten.
\item
  \textbf{Abtastrate}: Eine höhere Abtastrate ermöglicht eine genauere
  Darstellung schneller Signale.
\item
  \textbf{Kanäle}: Überlegen Sie, wie viele Signale Sie gleichzeitig
  messen müssen. Zwei Kanäle sind üblich, aber für komplexere Analysen
  könnten vier oder mehr Kanäle nötig sein.
\item
  \textbf{Speichertiefe}: Eine größere Speichertiefe ermöglicht es
  Ihnen, längere Zeiträume des Signals bei hoher Abtastrate zu
  speichern.
\item
  \textbf{Benutzerfreundlichkeit}: Die Bedienoberfläche sollte intuitiv
  sein, und das Gerät sollte über ausreichende Mess- und
  Analysefunktionen verfügen.
\item
  \textbf{Preis}: Oszilloskope reichen von günstigen Einsteigermodellen
  bis hin zu hochwertigen Geräten für professionelle Anwendungen.
\end{itemize}

\hypertarget{empfehlungen}{%
\subsubsection{Empfehlungen}\label{empfehlungen}}

\begin{enumerate}
\def\labelenumi{\arabic{enumi}.}

\item
  \textbf{Einsteiger}: Rigol DS1054Z

  \begin{itemize}
  
  \item
    Eine beliebte Wahl für Hobbyisten und Bildungszwecke. Es bietet eine
    gute Bandbreite (50 MHz, software-upgradebar) und vier Kanäle zu
    einem erschwinglichen Preis.
  \end{itemize}
\item
  \textbf{Mittleres Segment}: Siglent SDS1104X-E

  \begin{itemize}
  
  \item
    Ein Schritt nach oben mit einer Bandbreite von 100 MHz und vier
    Kanälen. Es bietet eine hohe Abtastrate und eine benutzerfreundliche
    Oberfläche, was es für ernsthafte Hobbyisten und semiprofessionelle
    Anwendungen geeignet macht.
  \end{itemize}
\item
  \textbf{Hochwertig/Professionell}: Keysight Technologies InfiniiVision
  3000T X-Series

  \begin{itemize}
  
  \item
    Für professionelle Anwender, die eine höhere Bandbreite (bis zu
    mehreren GHz), ausgezeichnete Abtastraten, tiefe Speichertiefe und
    fortschrittliche Analysefunktionen benötigen. Diese Oszilloskope
    sind teurer, bieten aber hervorragende Leistung und Zuverlässigkeit.
  \end{itemize}
\item
  \textbf{Portabel}: Fluke 123B Industrial ScopeMeter

  \begin{itemize}
  
  \item
    Wenn Mobilität wichtig ist, sind die ScopeMeter von Fluke robust und
    tragbar, ideal für Feldarbeiten oder wenn Sie ein Oszilloskop an
    verschiedenen Orten einsetzen müssen.
  \end{itemize}
\end{enumerate}

\hypertarget{siglent-sds1104x-e-oszilloskop}{%
\subsection{Siglent SDS1104X-E
Oszilloskop}\label{siglent-sds1104x-e-oszilloskop}}

Das Siglent SDS1104X-E ist ein digitales Speicheroszilloskop (DSO), das
sich durch eine hervorragende Mischung aus Leistung, Funktionen und
Preis auszeichnet, wodurch es sowohl für Hobbyisten als auch für
professionelle Anwender attraktiv ist.

\hypertarget{eigenschaften-des-siglent-sds1104x-e}{%
\subsubsection{Eigenschaften des Siglent
SDS1104X-E}\label{eigenschaften-des-siglent-sds1104x-e}}

\begin{itemize}

\item
  \textbf{Bandbreite}: 100 MHz, erweiterbar durch Software-Upgrade auf
  höhere Bandbreiten, was es flexibel für eine Vielzahl von Anwendungen
  macht.
\item
  \textbf{Kanäle}: Vier analoge Kanäle, die simultane Messungen
  ermöglichen und es ideal für komplexe Elektronikprojekte machen, bei
  denen mehrere Signale gleichzeitig verfolgt werden müssen.
\item
  \textbf{Abtastrate}: Bis zu 1 GSa/s (Giga-Samples pro Sekunde), was
  eine hohe Detailtreue bei der Darstellung schneller Signale
  ermöglicht.
\item
  \textbf{Speichertiefe}: 14 Mpts (Mega-Punkte) pro Kanal, was lange
  Aufzeichnungen bei hoher Abtastrate ermöglicht und somit eine
  detaillierte Analyse von Signalen über längere Zeiträume hinweg
  unterstützt.
\item
  \textbf{Display}: Ein großes 7-Zoll-TFT-LCD-Farbdisplay bietet eine
  klare und übersichtliche Anzeige der Messdaten.
\item
  \textbf{Messfunktionen}: Es bietet eine Vielzahl von automatisierten
  Messfunktionen sowie mathematische Funktionen wie FFT (Schnelle
  Fourier-Transformation) für die Signalanalyse.
\item
  \textbf{Konnektivität}: USB- und LAN-Schnittstellen für
  Datenübertragung und Fernsteuerung erweitern die Einsatzmöglichkeiten
  in verschiedenen Umgebungen.
\end{itemize}

\hypertarget{anwendungsmuxf6glichkeiten}{%
\subsubsection{Anwendungsmöglichkeiten}\label{anwendungsmoeglichkeiten}}

Mit dem Siglent SDS1104X-E können Sie eine breite Palette von Mess- und
Analyseaufgaben durchführen, darunter:

\begin{itemize}

\item
  \textbf{Signalprüfung}: Überprüfen Sie die Integrität und Qualität von
  analogen und digitalen Signalen.
\item
  \textbf{Fehlersuche}: Identifizieren Sie Probleme in elektronischen
  Schaltungen, von einfachen Geräten bis hin zu komplexen Systemen.
\item
  \textbf{Design und Entwicklung}: Testen und Verifizieren von
  elektronischen Designs während des Entwicklungsprozesses.
\item
  \textbf{Bildungsprojekte}: Ideal für Lehr- und Lernumgebungen, um die
  Grundlagen der Elektronik und Signalverarbeitung zu vermitteln.
\end{itemize}

Um das volle Potenzial Ihres Siglent SDS1104X-E Oszilloskops
auszuschöpfen, ist es wichtig, sich mit den Grundlagen vertraut zu
machen, die Voreinstellungen und automatischen Messfunktionen effektiv
zu nutzen und die mathematischen sowie Analysefunktionen zu erkunden.

\hypertarget{lernen-sie-die-grundlagen}{%
\subsubsection{1. Lernen Sie die
Grundlagen}\label{lernen-sie-die-grundlagen}}

\hypertarget{a.-aufbau-und-bedienelemente}{%
\paragraph{a. Aufbau und
Bedienelemente}\label{a.-aufbau-und-bedienelemente}}

\begin{itemize}

\item
  \textbf{Display}: Vertraut machen mit dem Layout des Displays,
  inklusive der Wellenformanzeige, Menüleiste und Statusanzeigen.
\item
  \textbf{Tastenfeld}: Lernen Sie die Funktionen der verschiedenen
  Tasten kennen, z.B. die vertikalen und horizontalen Skalierungstasten,
  die Menütasten für die Kanalauswahl und die Einstellungstasten für
  Trigger und Akquisition.
\item
  \textbf{Drehknöpfe}: Üben Sie die Verwendung der Drehknöpfe für die
  Einstellung der Signalanzeige, wie z.B. die Positionierung der
  Wellenform, die Feinjustierung der Skalierung und die
  Trigger-Niveau-Einstellung.
\end{itemize}

\hypertarget{b.-grundlegende-messungen}{%
\paragraph{b. Grundlegende Messungen}\label{b.-grundlegende-messungen}}

\begin{itemize}

\item
  \textbf{Kanalauswahl}: Aktivieren Sie einzelne Kanäle und passen Sie
  die Einstellungen wie Kopplung (AC/DC), Bandbreitenbegrenzung und
  Vertikalskalierung an.
\item
  \textbf{Triggerung}: Verstehen Sie, wie Sie die Triggerquelle und -art
  einstellen, um stabile Signalanzeigen zu erhalten. Experimentieren Sie
  mit verschiedenen Triggermodi wie Edge, Pulse und Video.
\item
  \textbf{Zeitbasis und Skalierung}: Lernen Sie, wie Sie die horizontale
  Skalierung anpassen, um verschiedene Teile des Signals über die Zeit
  zu betrachten.
\end{itemize}

\hypertarget{nutzen-sie-die-voreinstellungen-und-automatischen-messfunktionen}{%
\subsubsection{2. Nutzen Sie die Voreinstellungen und automatischen
Messfunktionen}\label{nutzen-sie-die-voreinstellungen-und-automatischen-messfunktionen}}

\hypertarget{a.-voreinstellungen-verwenden}{%
\paragraph{a. Voreinstellungen
verwenden}\label{a.-voreinstellungen-verwenden}}

\begin{itemize}

\item
  \textbf{Auto Set}: Nutzen Sie die „Auto Set>>-Taste, um automatisch
  eine geeignete Vertikal- und Horizontalskalierung für das angezeigte
  Signal zu wählen.
\item
  \textbf{Messvorlagen}: Verwenden Sie vordefinierte Messvorlagen für
  häufige Messaufgaben, um Zeit zu sparen.
\end{itemize}

\hypertarget{b.-automatische-messungen}{%
\paragraph{b. Automatische Messungen}\label{b.-automatische-messungen}}

\begin{itemize}

\item
  \textbf{Messfunktionen}: Zugriff auf die automatischen Messfunktionen
  für Parameter wie Frequenz, Amplitude, Anstiegszeit, Fallzeit und
  vieles mehr.
\item
  \textbf{Statistische Analyse}: Aktivieren Sie die statistische Analyse
  für Ihre Messungen, um Durchschnittswerte, Standardabweichungen und
  andere statistische Daten zu erhalten.
\end{itemize}

\hypertarget{experimentieren-sie-mit-den-mathematischen-und-analysefunktionen}{%
\subsubsection{3. Experimentieren Sie mit den mathematischen und
Analysefunktionen}\label{experimentieren-sie-mit-den-mathematischen-und-analysefunktionen}}

\hypertarget{a.-mathematische-operationen}{%
\paragraph{a. Mathematische
Operationen}\label{a.-mathematische-operationen}}

\begin{itemize}

\item
  \textbf{Grundlagen}: Nutzen Sie mathematische Operationen wie
  Addition, Subtraktion, Multiplikation und Division von Signalen, um
  komplexe Signale zu analysieren.
\item
  \textbf{FFT (Schnelle Fourier-Transformation)}: Verwenden Sie die
  FFT-Funktion, um das Frequenzspektrum Ihres Signals zu analysieren.
  Dies ist besonders nützlich für die Untersuchung von
  Signalverzerrungen, Harmonischen und Rauschen.
\end{itemize}

\hypertarget{b.-erweiterte-analyse}{%
\paragraph{b. Erweiterte Analyse}\label{b.-erweiterte-analyse}}

\begin{itemize}

\item
  \textbf{Signaldekodierung}: Wenn Ihr Modell es unterstützt, nutzen Sie
  die Signaldekodierungsfunktionen für serielle Busse wie I²C, SPI,
  UART/RS232, um Kommunikationsprotokolle zu analysieren.
\item
  \textbf{Wellenform-Aufzeichnung}: Experimentieren Sie mit der
  Aufzeichnung und dem Rückblick von Wellenformsequenzen, um transiente
  Ereignisse zu erfassen und zu analysieren.
\end{itemize}

\hypertarget{messen}{%
\subsection{Messen}\label{messen}}

Um Messungen mit einem Ultraschallsensor und einem IR-Sensor unter
Verwendung eines Oszilloskops durchzuführen, müssen Sie die spezifischen
Eigenschaften dieser Sensoren und die Art der Signale, die sie erzeugen
oder empfangen, berücksichtigen. Hier sind allgemeine Richtlinien für
beide Sensortypen:

\hypertarget{ultraschallsensor-messungen}{%
\subsubsection{Ultraschallsensor-Messungen}\label{ultraschallsensor-messungen}}

Ultraschallsensoren senden Schallwellen aus und empfangen das Echo
dieser Wellen, um die Entfernung zu Objekten zu messen. Typischerweise
werden Sie das Trigger-Signal (das den Sendevorgang startet) und das
Echo-Signal (das den Empfang des zurückkehrenden Schalls darstellt)
messen wollen.

\hypertarget{einstellungen-fuxfcr-das-oszilloskop-ultraschallsensoren}{%
\paragraph{Einstellungen für das Oszilloskop
Ultraschallsensoren}\label{einstellungen-fuer-das-oszilloskop-ultraschallsensoren}}

\begin{enumerate}
\def\labelenumi{\arabic{enumi}.}

\item
  \textbf{Kopplung}: Stellen Sie die Eingangskopplung auf DC, um sowohl
  den Gleichspannungs- als auch den Wechselspannungsanteil des Signals
  zu erfassen.
\item
  \textbf{Zeitbasis}: Wählen Sie eine Zeitbasis, die es Ihnen
  ermöglicht, den gesamten Vorgang des Sendens und Empfangens des Echos
  zu sehen. Ein guter Startpunkt könnte 1 ms/Div sein, abhängig von der
  erwarteten Reichweite des Sensors.
\item
  \textbf{Triggerung}: Verwenden Sie die Edge-Triggerung, getriggert auf
  das Trigger-Signal, um das Oszilloskop zu stabilisieren und das Senden
  des Signals sichtbar zu machen.
\item
  \textbf{Vertikalskalierung}: Passen Sie die Vertikalskalierung an, um
  die Amplitude des Trigger- und Echo-Signals klar zu sehen. Dies hängt
  von den Signalpegeln ab, die Ihr Sensor verwendet.
\end{enumerate}

\hypertarget{messungen-ultraschallsensoren}{%
\paragraph{Messungen
Ultraschallsensoren}\label{messungen-ultraschallsensoren}}

\begin{itemize}

\item
  \textbf{Trigger-Signal}: Überprüfen Sie die Dauer des Trigger-Signals,
  um sicherzustellen, dass es den Spezifikationen des Sensors entspricht
  (häufig 10 μs).
\item
  \textbf{Echo-Zeit}: Messen Sie die Zeit zwischen dem Ende des
  Trigger-Signals und dem Beginn des Echo-Signals, um die Entfernung zu
  berechnen. Die Echo-Zeit ist proportional zur Entfernung zum Objekt.
\item
  \textbf{Signalform}: Beobachten Sie die Form des Echo-Signals, um die
  Integrität und Stärke des empfangenen Signals zu bewerten.
\end{itemize}

\hypertarget{ir-sensor-messungen}{%
\subsubsection{IR-Sensor-Messungen}\label{ir-sensor-messungen}}

IR-Sensoren (Infrarotsensoren) werden oft für Näherungsmessungen,
Objekterkennung oder als Teil eines Kommunikationssystems verwendet. Die
Messung kann die Überprüfung des Ausgangssignals in Reaktion auf ein
erkanntes Objekt oder die Übertragung/der Empfang von
IR-Kommunikationssignalen umfassen.

\hypertarget{einstellungen-fuxfcr-das-oszilloskop-ir-sensor}{%
\paragraph{Einstellungen für das Oszilloskop
IR-Sensor}\label{einstellungen-fuer-das-oszilloskop-ir-sensor}}

\begin{enumerate}
\def\labelenumi{\arabic{enumi}.}

\item
  \textbf{Kopplung}: DC-Kopplung, um das vollständige Signal zu
  erfassen.
\item
  \textbf{Zeitbasis}: Die Zeitbasis sollte auf die erwartete
  Signalgeschwindigkeit eingestellt werden. Für eine kontinuierliche
  Signalüberwachung kann eine breitere Zeitbasis (z.B. 100 ms/Div)
  nützlich sein, während für die Untersuchung von Kommunikationssignalen
  eine engere Zeitbasis (z.B. 10 μs/Div) erforderlich sein kann.
\item
  \textbf{Triggerung}: Verwenden Sie die Edge-Triggerung, ausgelöst
  durch das IR-Signal, um das Oszilloskop zu stabilisieren.
\item
  \textbf{Vertikalskalierung}: Passen Sie diese so an, dass das
  IR-Signal deutlich sichtbar ist, je nach Ausgangspegel des Sensors.
\end{enumerate}

\hypertarget{messungen-ir-sensoren}{%
\paragraph{Messungen IR-Sensoren}\label{messungen-ir-sensoren}}

\begin{itemize}

\item
  \textbf{Signalpräsenz}: Bestätigen Sie die Präsenz des IR-Signals,
  wenn ein Objekt erkannt wird oder eine IR-Kommunikation stattfindet.
\item
  \textbf{Signalform}: Beurteilen Sie die Form des IR-Signals, um die
  Qualität und Stärke der Detektion oder Kommunikation zu überprüfen.
\item
  \textbf{Kommunikationsdaten}: Bei der Messung von
  IR-Kommunikationssignalen, überprüfen Sie die Pulsbreiten, um die
  übertragenen Daten zu decodieren.
\end{itemize}

Um die in Ihrem Projekt verwendeten 6x Getriebemotoren und 18650
Batterien effektiv mit einem Oszilloskop zu messen und zu analysieren,
folgen Sie diesen detaillierten Anweisungen. Diese Anleitung hilft
Ihnen, wichtige Aspekte wie PWM-Signale, Spannungsschwankungen,
Motorstart- und Stoppzeiten sowie den Spannungsverlauf und die
Stromaufnahme der Batterie zu überwachen.

\hypertarget{x-getriebemotoren-messen}{%
\subsubsection{6x Getriebemotoren
messen}\label{x-getriebemotoren-messen}}

\hypertarget{pwm-signal}{%
\paragraph{PWM-Signal}\label{pwm-signal}}

\begin{enumerate}
\def\labelenumi{\arabic{enumi}.}

\item
  \textbf{Verbindung herstellen}: Verbinden Sie das Oszilloskop mit den
  PWM-Ausgängen, die die Motoren steuern. Verwenden Sie geeignete
  Tastköpfe für Ihre Messungen.
\item
  \textbf{Oszilloskop einstellen}: Stellen Sie das Oszilloskop auf eine
  geeignete Zeitbasis ein, um die PWM-Signale klar zu sehen,
  typischerweise einige Mikrosekunden pro Division (μs/div), abhängig
  von der PWM-Frequenz.
\item
  \textbf{Messung durchführen}: Beobachten Sie die Pulsbreite und
  Frequenz der PWM-Signale. Überprüfen Sie, ob diese mit den erwarteten
  Werten für die Steuerung der Motorgeschwindigkeit und -richtung
  übereinstimmen.
\end{enumerate}

\hypertarget{spannungsschwankungen}{%
\paragraph{Spannungsschwankungen}\label{spannungsschwankungen}}

\begin{enumerate}
\def\labelenumi{\arabic{enumi}.}

\item
  \textbf{Verbindung}: Verbinden Sie das Oszilloskop direkt mit den
  Motoranschlüssen, um die Spannung zu messen.
\item
  \textbf{Einstellungen anpassen}: Wählen Sie die AC-Kopplung, um
  Spannungsschwankungen, oder die DC-Kopplung, um die Gesamtspannung zu
  beobachten. Stellen Sie die Vertikalskalierung entsprechend der
  erwarteten Spannungshöhe ein.
\item
  \textbf{Beobachtung}: Achten Sie auf Spannungsspitzen oder -einbrüche,
  die auf Probleme in der Motorsteuerung oder Stromversorgung hinweisen
  könnten.
\end{enumerate}

\hypertarget{motorstart--und-stoppzeiten}{%
\paragraph{Motorstart- und
Stoppzeiten}\label{motorstart--und-stoppzeiten}}

\begin{enumerate}
\def\labelenumi{\arabic{enumi}.}

\item
  \textbf{Triggerung einstellen}: Verwenden Sie das PWM-Signal als
  Triggerquelle, um das Oszilloskop zu stabilisieren.
\item
  \textbf{Zeit messen}: Messen Sie die Zeit zwischen dem Startsignal
  (Änderung des PWM-Signals) und der sichtbaren Reaktion des Motors.
  Verwenden Sie die Cursor-Funktion des Oszilloskops für eine präzise
  Messung.
\end{enumerate}

\hypertarget{batterie-messen}{%
\subsubsection{18650 Batterie messen}\label{batterie-messen}}

\hypertarget{spannungsverlauf}{%
\paragraph{Spannungsverlauf}\label{spannungsverlauf}}

\begin{enumerate}
\def\labelenumi{\arabic{enumi}.}

\item
  \textbf{Verbindung}: Schließen Sie das Oszilloskop direkt an die
  Batteriepole an.
\item
  \textbf{Unter Last messen}: Aktivieren Sie den Rover, um verschiedene
  Aktionen durchzuführen, und beobachten Sie die Spannung der Batterie
  unter Last.
\item
  \textbf{Analyse}: Achten Sie auf signifikante Spannungseinbrüche oder
  -spitzen, die auf eine unzureichende Batterieleistung oder potenzielle
  Probleme in der Stromversorgung hinweisen könnten.
\end{enumerate}

\hypertarget{stromaufnahme}{%
\paragraph{Stromaufnahme}\label{stromaufnahme}}

\begin{enumerate}
\def\labelenumi{\arabic{enumi}.}

\item
  \textbf{Shunt-Widerstand}: Schalten Sie einen geeigneten
  Shunt-Widerstand in Serie zum Stromkreis der Batterie.
\item
  \textbf{Verbindung}: Verbinden Sie das Oszilloskop über den
  Shunt-Widerstand, um die Spannung über diesem zu messen. Die gemessene
  Spannung ist proportional zum Strom durch den Shunt-Widerstand.
\item
  \textbf{Berechnung}: Berechnen Sie den Strom durch den
  Shunt-Widerstand anhand des Ohmschen Gesetzes (I = V/R), wobei V die
  gemessene Spannung über dem Shunt und R der Widerstandswert des
  Shunt-Widerstands ist.
\end{enumerate}

\hypertarget{zusuxe4tzliche-tipps}{%
\subsubsection{Zusätzliche Tipps}\label{zusaetzliche-tipps}}

\begin{itemize}

\item
  \textbf{Sicherheit}: Achten Sie stets auf die Sicherheit beim Umgang
  mit elektrischen Komponenten, insbesondere bei der Arbeit mit
  Batterien und Motoren.
\item
  \textbf{Dokumentation}: Halten Sie die Ergebnisse Ihrer Messungen
  fest, um Veränderungen über die Zeit oder nach Anpassungen im System
  zu vergleichen.
\end{itemize}

\hypertarget{shunt-widerstands-fuxfcr-die-strommessung}{%
\subsection{Shunt-Widerstands für die
Strommessung}\label{shunt-widerstands-fuer-die-strommessung}}

Die Wahl eines geeigneten Shunt-Widerstands für die Strommessung in
Ihrem Rover-Projekt hängt von mehreren Faktoren ab, einschließlich des
maximalen Stroms, der durch das System fließt, der Spannung, die Ihr
Messgerät sicher handhaben kann, und der Auflösung, die Sie erreichen
möchten. Hier sind einige Überlegungen und Empfehlungen:

\hypertarget{uxfcberlegungen}{%
\subsubsection{Überlegungen}\label{ueberlegungen}}

\begin{enumerate}
\def\labelenumi{\arabic{enumi}.}
\item
  \textbf{Maximaler Strom}: Bestimmen Sie den maximalen Strom, der durch
  Ihren Rover fließen wird. Dieser Wert ist wichtig, um einen
  Shunt-Widerstand mit einer geeigneten Nennleistung zu wählen.
\item
  \textbf{Spannungsabfall}: Der Shunt-Widerstand erzeugt einen
  Spannungsabfall proportional zum durchfließenden Strom. Dieser
  Spannungsabfall sollte groß genug sein, um messbar zu sein, aber nicht
  so groß, dass er die Leistung Ihres Systems beeinträchtigt. Ein
  typischer Wert für den Spannungsabfall über einen Shunt-Widerstand
  liegt zwischen 50 mV und 200 mV bei maximalem Strom.
\item
  \textbf{Auflösung und Genauigkeit}: Die Auflösung Ihres Messgeräts
  (z.B. eines Oszilloskops oder eines Multimeters im Strommessmodus)
  bestimmt, wie klein der Widerstandswert des Shunt sein kann, ohne dass
  die Messgenauigkeit beeinträchtigt wird.
\end{enumerate}

\hypertarget{empfehlungen-shunt-widerstands-fuxfcr-die-strommessung}{%
\subsubsection{Empfehlungen Shunt-Widerstands für die
Strommessung}\label{empfehlungen-shunt-widerstands-fuer-die-strommessung}}

\begin{itemize}
\item
  \textbf{Für niedrige Ströme (bis zu 1 A)}: Ein Widerstandswert von 0,1
  Ohm bis 1 Ohm mit einer Nennleistung von mindestens 1 W könnte
  geeignet sein. Dies würde einen messbaren Spannungsabfall erzeugen,
  ohne zu viel Leistung zu verbrauchen.
\item
  \textbf{Für mittlere Ströme (1 A bis 10 A)}: Ein Widerstandswert im
  Bereich von 10 mOhm bis 100 mOhm mit einer Nennleistung von 2 W bis 5
  W könnte passend sein. Für solche Anwendungen sind
  Präzisions-Shunt-Widerstände mit niedrigem Temperaturkoeffizienten und
  hoher Genauigkeit verfügbar.
\item
  \textbf{Für höhere Ströme (über 10 A)}: Sie benötigen möglicherweise
  einen spezialisierten Shunt-Widerstand mit einem sehr niedrigen
  Widerstandswert (z.B. 1 mOhm bis 10 mOhm) und einer hohen Nennleistung
  (10 W oder mehr), um die Wärmeentwicklung zu bewältigen.
\end{itemize}

\hypertarget{spezifisches-produkt}{%
\subsubsection{Spezifisches Produkt}\label{spezifisches-produkt}}

Ein spezifisches Produkt zu empfehlen, ohne genauere Details zu Ihrem
Projekt zu kennen, ist schwierig. Allerdings sind Shunt-Widerstände von
Herstellern wie Vishay, Bourns oder TE Connectivity allgemein als
hochwertig anerkannt. Ein Beispiel für einen häufig verwendeten
Shunt-Widerstand ist der Vishay Dale WSBS8518\ldots{} Serie für höhere
Stromanwendungen, der eine Nennleistung von bis zu 36 W aufweisen kann
und für Strommessungen in Automotive- und Industrieanwendungen
entwickelt wurde.

\hypertarget{verhalten-des-rovers-unter-realen-betriebsbedingungen-zu-analysieren}{%
\subsection{Verhalten des Rovers unter realen Betriebsbedingungen zu
analysieren}\label{verhalten-des-rovers-unter-realen-betriebsbedingungen-zu-analysieren}}

Für die Analyse des Verhaltens Ihres Rovers unter realen
Betriebsbedingungen würde ich ein integriertes System aus Datenloggern
und drahtloser Kommunikation empfehlen, das es ermöglicht, Daten in
Echtzeit zu erfassen und zu überwachen, während der Rover sich bewegt.

\hypertarget{datenerfassung-und--speicherung}{%
\subsubsection{Datenerfassung und
-speicherung}\label{datenerfassung-und--speicherung}}

\begin{itemize}

\item
  \textbf{Onboard-Datenlogger}: Verwenden Sie SD-Kartenmodule für
  Arduino oder die SD-Kartenfunktion des Raspberry Pi, um Daten zu
  speichern, falls die drahtlose Verbindung unterbrochen wird.
\item
  \textbf{Datenformatierung}: Implementieren Sie eine
  Datenspeicherstruktur, die Zeitstempel für jede Messung enthält, um
  die Analyse zu erleichtern.
\end{itemize}

\hypertarget{esp32-cam}{%
\subsubsection{ESP32-CAM}\label{esp32-cam}}

\hypertarget{echtzeit-videouxfcberwachung}{%
\paragraph{Echtzeit-Videoüberwachung}\label{echtzeit-videoueberwachung}}

\begin{itemize}

\item
  \textbf{Kamera}: Nutzen Sie die integrierte OV2640 Kamera des
  ESP32-CAM für Echtzeit-Videoüberwachung oder zur Aufnahme von Fotos.
  Dies ermöglicht Ihnen, die Umgebung, in der sich der Rover bewegt,
  visuell zu analysieren und zu dokumentieren.
\item
  \textbf{Video Streaming}: Implementieren Sie Video-Streaming über
  Wi-Fi, um die Live-Bilder vom Rover auf ein Smartphone, Tablet oder
  einen PC zu übertragen. Dies kann besonders nützlich sein, um die
  Navigation des Rovers zu überwachen und zu steuern.
\end{itemize}

\hypertarget{datenerfassung-und--uxfcbertragung}{%
\paragraph{Datenerfassung und
-übertragung}\label{datenerfassung-und--uebertragung}}

\begin{itemize}

\item
  \textbf{Sensordaten}: Neben der Videodaten können Sie auch andere
  Sensordaten wie Temperatur, Beschleunigung oder sogar die Position
  (über GPS, falls angeschlossen) erfassen.
\item
  \textbf{Drahtlose Kommunikation}: Verwenden Sie Wi-Fi oder Bluetooth
  für die drahtlose Übertragung dieser Daten in Echtzeit oder speichern
  Sie die Daten auf einer microSD-Karte für spätere Analysen.
\end{itemize}

\hypertarget{erweiterte-navigation-und-steuerung}{%
\paragraph{Erweiterte Navigation und
Steuerung}\label{erweiterte-navigation-und-steuerung}}

\begin{itemize}

\item
  \textbf{Integration des geomagnetischen Chips QMC6310}: Nutzen Sie den
  geomagnetischen Sensor für fortschrittliche Navigationsfunktionen, wie
  die Bestimmung der Ausrichtung des Rovers. Dies kann für
  automatisierte Navigationsrouten oder zur Verbesserung der
  Wegfindungsalgorithmen hilfreich sein.
\item
  \textbf{Bewegungserkennung und -analyse}: Analysieren Sie die Bewegung
  des Rovers durch die Videoaufnahmen, um z.B. die Effizienz der
  Wegfindung oder das Verhalten bei der Hindernisumgehung zu beurteilen.
\end{itemize}

\hypertarget{programmierung-und-anpassung}{%
\paragraph{Programmierung und
Anpassung}\label{programmierung-und-anpassung}}

\begin{itemize}

\item
  \textbf{Anpassbare Skripte}: Entwickeln Sie eigene Skripte für die
  ESP32-CAM, um spezifische Aufgaben wie das Auslösen von Fotos unter
  bestimmten Bedingungen, die Steuerung der Bewegung basierend auf
  visuellen Signalen oder die Anpassung der Datenübertragungsraten zu
  automatisieren.
\end{itemize}

\hypertarget{sicherheits--und-uxfcberwachungsanwendungen}{%
\paragraph{Sicherheits- und
Überwachungsanwendungen}\label{sicherheits--und-ueberwachungsanwendungen}}

\begin{itemize}

\item
  \textbf{Überwachung}: Setzen Sie den Rover für
  Sicherheitsüberwachungsaufgaben ein, indem Sie die Kamera nutzen, um
  Bereiche zu überwachen und bei Erkennung von Bewegungen oder anderen
  definierten Ereignissen Benachrichtigungen zu senden.
\end{itemize}

\hypertarget{hinweise-zur-implementierung}{%
\paragraph{Hinweise zur
Implementierung}\label{hinweise-zur-implementierung}}

\begin{itemize}

\item
  Stellen Sie sicher, dass die Stromversorgung des ESP32-CAM Moduls den
  Anforderungen entspricht, insbesondere wenn Sie planen, es
  kontinuierlich für Video-Streaming oder umfangreiche Datenerfassung zu
  nutzen.
\item
  Beachten Sie die IO-Pinbelegung und die Funktionen der Pins,
  insbesondere wenn Sie zusätzliche Sensoren oder Module anschließen
  möchten.
\end{itemize}

Das ESP32-CAM Modul bietet eine kostengünstige und flexible Lösung für
erweiterte Rover-Projekte, die Videoüberwachung und die Erfassung von
Sensordaten in Echtzeit erfordern. Mit seiner Hilfe können Sie nicht nur
das Verhalten und die Leistung Ihres Rovers unter verschiedenen
Bedingungen analysieren, sondern auch innovative Funktionen für
Navigation, Sicherheit und Interaktion implementieren. % Platzhalter

%% Optional Anhang
%\clearpage
%\appendix

\clearpage
\printbibliography
\end{document}