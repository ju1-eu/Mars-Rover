% ju 19-Feb-24 README.tex
\documentclass{vorlage-design-main}
\usepackage[utf8]{inputenc}
\usepackage{longtable}
\usepackage{blindtext,alltt}
%% Ganze Überschrift
\title{Thema}

%% Kürzerer Titel zur Verwendung im Seitenkopf
\runningtitle{Kurztitel}
\author{Jan Unger}
% \author{2.}
\date{\today}

%% Die .bib-Datei mit vollständigen Referenzen zur Verwendung mit biblatex. articleclass lädt das Paket biblatex-chicago mit Anpassungen
\addbibresource{literatur.bib}

\begin{document}

\maketitle

\begin{abstract}

\end{abstract}

\hypertarget{diy-projekte}{%
\subsection{DIY-Projekte}\label{diy-projekte}}

>>DIY<< steht für >>Do It Yourself<< und bezieht sich auf Projekte, bei
denen Individuen eigenständig Dinge kreieren, bauen, modifizieren oder
reparieren, anstatt sie fertig zu kaufen oder von Fachleuten erledigen
zu lassen. DIY-Projekte können eine breite Palette von Aktivitäten
umfassen, von Heimwerkerarbeiten über Elektronik und
Softwareentwicklung. Mit der Verfügbarkeit von Anleitungen und
Ressourcen im Internet hat die DIY-Bewegung an Popularität gewonnen,
wobei Menschen weltweit Projekte teilen und voneinander lernen.

\hypertarget{iot-internet-of-things}{%
\subsection{IoT (Internet of Things)}\label{iot-internet-of-things}}

IoT, das >>Internet der Dinge<<, bezieht sich auf die Vernetzung von
physischen Objekten mit dem Internet, wodurch diese Objekte Daten
sammeln, senden und auf Anweisungen reagieren können. Diese Objekte, oft
als >>smart<< oder >>intelligent<< bezeichnet, können eine breite
Palette von Geräten und Sensoren umfassen, wie z.B. Beleuchtungssysteme,
Umweltsensoren und Fahrzeuge.

\hypertarget{steam-bildung}{%
\subsection{STEAM-Bildung}\label{steam-bildung}}

STEAM-Bildung steht für einen interdisziplinären Ansatz in der Bildung,
der die Integration von Wissenschaft (Science), Technologie
(Technology), Ingenieurwesen (Engineering), Kunst (Arts) und Mathematik
(Mathematics) betont. Dieses Bildungsmodell zielt darauf ab, Schülern
ein umfassendes Verständnis und Fähigkeiten in diesen Bereichen zu
vermitteln, indem es die Grenzen zwischen den traditionell getrennten
Fächern überbrückt. Der Ansatz fördert kritisches Denken,
Problemlösungsfähigkeiten, Kreativität und Innovation bei den Lernenden.

\hypertarget{hauptmerkmale-der-steam-bildung}{%
\subsubsection{Hauptmerkmale der
STEAM-Bildung}\label{hauptmerkmale-der-steam-bildung}}

\begin{enumerate}
\def\labelenumi{\arabic{enumi}.}
\item
  \textbf{Interdisziplinärer Ansatz}: STEAM integriert Konzepte und
  Fähigkeiten aus verschiedenen Disziplinen, um Lernenden zu helfen,
  Verbindungen zwischen den Fächern zu sehen und wie diese in der realen
  Welt angewendet werden können.
\item
  \textbf{Projektbasiertes Lernen}: STEAM fördert oft projektbasiertes
  Lernen, bei dem Schüler an Projekten arbeiten, die reale Probleme
  lösen oder kreative Werke schaffen. Dieser Ansatz fördert Teamarbeit,
  Kommunikationsfähigkeiten und praktische Erfahrungen.
\item
  \textbf{Förderung der Kreativität und Innovation}: Durch die
  Einbeziehung der Kunst in den traditionellen STEM-Ansatz (ohne >>A<<
  für Arts) werden Schüler ermutigt, kreativ und innovativ in ihrem
  Denken zu sein, was zu neuen Ideen und Lösungen führt.
\item
  \textbf{Betonung des lebenslangen Lernens}: STEAM-Bildung zielt darauf
  ab, die Neugier und die Liebe zum Lernen bei Schülern zu wecken, indem
  sie zeigt, wie das Gelernte in alltäglichen Situationen und
  zukünftigen Karrieren angewendet werden kann.
\item
  \textbf{Zugänglichkeit und Diversität}: STEAM-Programme bemühen sich,
  für alle Schüler unabhängig von ihrem Hintergrund zugänglich zu sein
  und fördern die Diversität in den Bereichen Wissenschaft, Technologie,
  Ingenieurwesen, Kunst und Mathematik.
\end{enumerate}

\hypertarget{gnu-general-public-license-gpl}{%
\subsection{GNU General Public License
(GPL)}\label{gnu-general-public-license-gpl}}

Die GNU General Public License (GPL) ist eine freie Softwarelizenz, die
das Recht gewährt, Software frei zu verwenden, zu studieren, zu ändern
und zu verbreiten. Entwickelt von der Free Software Foundation (FSF),
zielt die GPL darauf ab, die Freiheit zu sichern, Software zu teilen und
zu ändern, um sicherzustellen, dass Software für alle ihre Benutzer frei
ist. Es gibt mehrere Versionen der GPL, wobei die Version 3 (GPLv3) die
neueste ist.

Die Kernprinzipien der GPL umfassen:

\begin{enumerate}
\def\labelenumi{\arabic{enumi}.}

\item
  \textbf{Freiheit zur Ausführung der Software für jeden Zweck}:
  Benutzer dürfen die Software in jeder Umgebung, für jeden Zweck
  nutzen.
\item
  \textbf{Freiheit zum Studium, wie die Software funktioniert, und sie
  an die eigenen Bedürfnisse anzupassen}: Der Zugang zum Quellcode ist
  eine Voraussetzung.
\item
  \textbf{Freiheit zur Weiterverbreitung von Kopien}: Benutzer können
  Kopien der Software frei verteilen.
\item
  \textbf{Freiheit zur Verbesserung der Software und zur
  Veröffentlichung der eigenen Verbesserungen (und modifizierten
  Versionen im Allgemeinen)}, sodass die gesamte Community davon
  profitiert: Zugang zum Quellcode ist hierfür ebenfalls erforderlich.
\end{enumerate}

\hypertarget{software-uxe4ndern-die-unter-gpl-lizenziert-ist}{%
\subsubsection{Software ändern die unter GPL lizenziert
ist}\label{software-aendern-die-unter-gpl-lizenziert-ist}}

Wenn Sie Software ändern, die unter der GNU General Public License (GPL)
lizenziert ist, gibt es bestimmte Anforderungen, die Sie erfüllen
müssen, um den Bedingungen der Lizenz gerecht zu werden. Hier sind die
wichtigsten Punkte, die Sie beachten sollten:

\begin{enumerate}
\def\labelenumi{\arabic{enumi}.}
\item
  \textbf{Quellcode muss verfügbar bleiben}: Wenn Sie die Software
  ändern und die modifizierte Version verbreiten, müssen Sie auch den
  modifizierten Quellcode unter der GPL zur Verfügung stellen. Dies
  stellt sicher, dass Empfänger der modifizierten Software dieselben
  Freiheiten haben, die ursprüngliche Software zu verwenden, zu
  studieren, zu ändern und weiterzuverteilen.
\item
  \textbf{Lizenzierung unter der GPL}: Änderungen an der Software müssen
  ebenfalls unter der GPL lizenziert werden. Dies bedeutet, dass Sie
  keine zusätzlichen Einschränkungen auferlegen dürfen, die über die
  Anforderungen der GPL hinausgehen.
\item
  \textbf{Vermerk von Änderungen}: Es ist eine gute Praxis (und in
  manchen Fällen erforderlich), in den Dateien, die Sie geändert haben,
  einen Vermerk über die Änderungen einzufügen, einschließlich des
  Datums der Änderung. Dies erhöht die Transparenz und hilft anderen
  Nutzern und Entwicklern, die Geschichte und die Art der durchgeführten
  Änderungen zu verstehen.

  \begin{itemize}
  
  \item
    2024-02-16, Max Mustermann: Funktion zur Berechnung der
    Geschwindigkeit hinzugefügt.
  \item
    2024-02-20, Erika Musterfrau: Fehler in der
    Geschwindigkeitsberechnung behoben.
  \end{itemize}
\item
  \textbf{Beibehaltung von Urheberrechts- und Lizenzhinweisen}: Sie
  müssen alle Urheberrechts- und Lizenzhinweise, die in der
  ursprünglichen Software enthalten sind, unverändert lassen. Wenn Sie
  neue Dateien hinzufügen, die von Ihnen erstellt wurden, können Sie in
  diesen Dateien eigene Urheberrechts- und Lizenzhinweise hinzufügen,
  müssen jedoch sicherstellen, dass Ihre Lizenzierung konsistent mit der
  GPL ist.
\end{enumerate}

\newpage

\hypertarget{projekt-erstellen}{%
\subsection{Projekt erstellen}\label{projekt-erstellen}}

\begin{lstlisting}[language=bash]
# Projekt erstellen
cd /Users/jan/daten/start/IOT/Mars-Rover

# Verzeichnisse erstellen
mkdir Beispiele Tutorials Dokumentation Projekte

# Notizen klonen und vorbereiten
git clone /Users/jan/notizen_latex_html_python_v1.git Notizen
cd Notizen
rm -rf .git
cd ..

# Initialisiere das Git-Repository
git init

# Erstelle eine neue Datei
touch TODO.md

# Füge Dateien zum Repository hinzu
git add .
# Committe die Änderungen
git commit -m "Initialer Commit"

# Remote-Repository auf GitHub erstellen
gh repo create Mars-Rover --public
# Füge das Remote-Repository hinzu (nachdem es auf GitHub erstellt wurde)
git remote add origin https://github.com/ju1-eu/Mars-Rover.git

# Push Änderungen zum Remote-Repository
git push -u origin main
# Änderungen von GitHub pullen (wenn nötig)
git pull origin main

# Personal Access Token (PAT) auf GitHub:
# https://github.com/settings/tokens
# 'admin:org', 'delete_repo', 'project', 'repo', 'workflow', 'write:discussion'
# Klicke auf "Generate new token"
gh auth status
gh auth login -h github.com

#---------------------------------
git add .
git commit -m"update"
git st
git push
git pull

git lg
git remote -v
git branch -a
#---------------------------------

# Lokales Bare-Repository erstellen: /Users/jan/Mars-Rover.git
git init --bare /Users/jan/Mars-Rover.git
# Verbinde dein bestehendes Projekt mit dem lokalen Bare-Repository
git remote add local-bare /Users/jan/Mars-Rover.git
# Push Änderungen zum lokalen Bare-Repository
git push local-bare main

# Lokales Repository für die Arbeit klonen
git clone /Users/jan/Mars-Rover.git
# Remote-Repository von GitHub klonen
git clone https://github.com/ju1-eu/Mars-Rover.git
\end{lstlisting}

\hypertarget{platformio}{%
\subsection{PlatformIO}\label{platformio}}

\begin{lstlisting}[language=bash]
# VS-Code / PlatformIO / Terminal
# PlatformIO CLI verwenden, um Arduino - Bibliotheken zu installieren
pio lib install "Servo"
pio lib install "SoftPWM"
# Cache löschen
pio run -t clean

# Projektordner verschieben
cd /Users/jan/Documents/PlatformIO/Projects
mv elektrik_test /Users/jan/PlatformIO/Projects
# VS-Code / Arbeitsbereich speichern unter / elektrik_test.code-workspace + Alias erstellen

# Erstellen eines Backups
cd /Users/jan/PlatformIO/Projects
zip -r Projekt_Backup_elektrik_test.zip elektrik_test/
mv *.zip /Users/jan/daten/start/IoT/Mars-Rover/Projekte
\end{lstlisting}

\hypertarget{chatgpt}{%
\subsection{ChatGPT}\label{chatgpt}}

ChatGPT \url{https://chat.openai.com/}

\begin{lstlisting}
# Keywords
Erstelle Schlüsselbegriffe und ihre Bedeutungen für das Verständnis der Funktionsweise des Codes

# Flussdiagramm für eine Programmlogik
"Ich hätte gerne ein Flussdiagramm in ASCII-Art (oder als Beschreibung für eine grafische Darstellung), das die Hauptlogik und den Datenfluss meines Programms verdeutlicht. Besonders interessiert mich, wie Entscheidungen getroffen werden und wie Daten zwischen den Funktionen fließen. Das Programm beinhaltet Funktionen zur Datenverarbeitung, Fehlerbehandlung und Benutzerinteraktion. Könnten Sie bitte die Schlüsselabläufe visualisieren, insbesondere wie das Programm auf Eingaben reagiert und wie es zu verschiedenen Ausgaben kommt? Ich bin auch an der Darstellung von Schleifen und bedingten Anweisungen interessiert. Falls nötig, hier sind einige Begriffe, die in meinem Programm verwendet werden: [...Begriffe einfügen]."

# Beschreibung für eine Programmlogik
"Ich hätte gerne eine Beschreibung, das die Hauptlogik und den Datenfluss meines Programms verdeutlicht. Besonders interessiert mich, wie Entscheidungen getroffen werden und wie Daten zwischen den Funktionen fließen. Das Programm beinhaltet Funktionen zur Datenverarbeitung, Fehlerbehandlung und Benutzerinteraktion. Könnten Sie bitte die Schlüsselabläufe visualisieren, insbesondere wie das Programm auf Eingaben reagiert und wie es zu verschiedenen Ausgaben kommt? Ich bin auch an der Darstellung von Schleifen und bedingten Anweisungen interessiert. Falls nötig, hier sind einige Begriffe, die in meinem Programm verwendet werden: [...Begriffe einfügen]."

# Für C- oder C++-Projekte wird oft der JavaDoc-Stil empfohlen
Erstelle Code mit Doxygen-kompatiblen Kommentaren

# Zusammenfassung
benutze Konventionen für Markdown
Schreibstil: Expositorisch ohne Form du/sie
Erstelle je nach Schreibstil eine ansprechende Zusammenfassung des folgenden Artikels in Aufzählungsform und gleichzeitig gebe die wichtigsten Informationen (Didaktische Reduktion) genau wieder. Bereite die Antwort gehirngerecht auf.

# ESP32 für PlattformIO programmieren
# Konventionen von PlatformIO für C++-Projekte
Erstelle ein "Hallo Welt"-C++Programm für einen ESP32 Mikrocontroller 
Erstelle ein "Hallo Welt"-Programm in MicroPython für einen ESP32 Mikrocontroller 
\end{lstlisting}

\hypertarget{quellen}{%
\subsection{Quellen}\label{quellen}}

\href{https://docs.sunfounder.com/projects/galaxy-rvr/de/latest/index.html}{GalaxyRVR
- SunFounder Mars Rover Kit}

Quellcode \url{https://github.com/sunfounder/galaxy-rvr}

Dokumentation
\url{https://github.com/sunfounder/galaxy-rvr/blob/docs-de/docs/source/index.rst}

Arduino Language Reference \url{https://www.arduino.cc/reference/en/}

Arduino Downloads \url{https://www.arduino.cc/en/software}

\textbf{Sonix} \url{https://sonix.ai/de/fast-transcription}

Sonix ist ein automatisierter Transkriptionsdienst, der speziell für die
Genauigkeit bei der Transkription von Audio- und Videodateien entwickelt
wurde. Sonix unterstützt mehrere Sprachen und bietet eine einfache
Möglichkeit, YouTube-Videos zu transkribieren, indem du die URL des
Videos hochlädst.

\begin{enumerate}
\def\labelenumi{\arabic{enumi}.}

\item
  Lektion 1: Enthüllung des Mars-Rovers
\item
  Lektion 2: Verständnis und Bau des Rocker-Bogie-Systems
\item
  Lektion 3: Einstieg in die Welt von Arduino und Programmierung
\item
  Lektion 4: Beherrschung des TT-Motors
\item
  Lektion 5: Entfesselung der Beweglichkeit des Mars Rovers
\item
  Lektion 6: Erkundung des Hindernisvermeidungsmoduls
\item
  Lektion 7: Verbesserung der Rover-Navigation mit Ultraschallmodul
\item
  Lektion 8 Fortgeschrittene Hindernisvermeidung und intelligentes
  Folgesystem
\item
  Lektion 9: Den Weg mit RGB-LED-Streifen beleuchten
\item
  Lektion 10: Erkundung des visuellen Systems des Mars-Rovers - Servo
  und Kippmechanismus
\item
  Lektion 11: Erforschung des visuellen Systems des Mars-Rovers - Kamera
  und Echtzeitsteuerung
\item
  Lektion 12: Steuerung des Rovers mit der App
\item
  Lektion 13: Untersuchung des Energiesystems des Mars-Rovers
\end{enumerate}

\hypertarget{hardware}{%
\subsection{Hardware}\label{hardware}}

\begin{itemize}

\item
  SunFounder R3 Board
\item
  GalaxyRVR Shield
\item
  ESP32 CAM
\item
  Kamera-Adapterplatine
\item
  Ultraschallmodul
\item
  IR-Hindernisvermeidungsmodul
\item
  4 RGB-LED-Streifen
\item
  Servo
\item
  TT-Motor
\item
  Solarpanel
\item
  18650 Batterie
\end{itemize}

\hypertarget{faq}{%
\subsection{FAQ}\label{faq}}

\begin{enumerate}
\def\labelenumi{\arabic{enumi}.}

\item
  Q1: Kompilierungsfehler: SoftPWM.h: Datei oder Verzeichnis nicht
  gefunden?
\item
  Q2: avrdude: stk500\_getsync() Versuch 10 von 10: nicht
  synchronisiert: resp=0x6e?
\item
  Q3: Wie kann ich den STT-Modus auf meinem Android-Gerät verwenden?
\item
  Q4: Über die ESP32 CAM Firmware
\item
  Q5: Wie wird eine neue Firmware auf einen ESP32 CAM geflasht?
\end{enumerate} % Platzhalter

%% Optional Anhang
%\clearpage
%\appendix

\clearpage
\printbibliography
\end{document}