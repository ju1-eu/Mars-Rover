% ju 19-Feb-24 Ultraschallmodul.tex
\documentclass{vorlage-design-main}
\usepackage[utf8]{inputenc}
\usepackage{longtable}
\usepackage{blindtext,alltt}
%% Ganze Überschrift
\title{Thema}

%% Kürzerer Titel zur Verwendung im Seitenkopf
\runningtitle{Kurztitel}
\author{Jan Unger}
% \author{2.}
\date{\today}

%% Die .bib-Datei mit vollständigen Referenzen zur Verwendung mit biblatex. articleclass lädt das Paket biblatex-chicago mit Anpassungen
\addbibresource{literatur.bib}

\begin{document}

\maketitle

\begin{abstract}

\end{abstract}

\begin{itemize}

\item
  \textbf{Einleitung in die Ultraschallnavigation}

  \begin{itemize}
  
  \item
    Ein neues >>Seh<<-System für den Mars-Rover wird eingeführt, um
    dessen Fähigkeit zur Hinderniserkennung zu verbessern, indem ein
    Ultraschallsensormodul hinzugefügt wird.
  \item
    Das Ziel ist, die Wahrnehmung des Rovers für Hindernisse direkt vor
    ihm zu schärfen, um die Navigation in komplexem Gelände zu
    optimieren.
  \end{itemize}
\item
  \textbf{Ultraschallsensor}

  \begin{itemize}
  
  \item
    Der HC-SR04 Ultraschallsensor kann Entfernungen von 2 cm bis 400 cm
    messen und ermöglicht dem Rover, seine Umgebung berührungslos zu
    >>sehen<<.
  \item
    Die Funktionsweise des Sensors ähnelt der Echolokation einer
    Fledermaus, nutzt Schallwellen zur Entfernungsmessung und verbessert
    so die Navigation des Rovers.
  \end{itemize}
\item
  \textbf{Funktionsprinzip des Ultraschallsensors}

  \begin{itemize}
  
  \item
    Der Sensor verwendet vier Pins (TRIG, ECHO, VCC, GND) für seine
    Funktion: TRIG sendet Schallwellen aus, ECHO empfängt das Echo der
    Wellen, VCC versorgt den Sensor mit Strom, und GND dient als Erdung.
  \item
    Die Entfernungsmessung erfolgt durch Aussenden und Empfangen von
    Ultraschallwellen, wobei die Zeit zwischen Aussendung und
    Echoempfang zur Berechnung der Entfernung zu Objekten genutzt wird.
  \end{itemize}
\end{itemize}

\hypertarget{berechnung}{%
\subsection{Berechnung}\label{berechnung}}

Um die Dauer des Echos zu berechnen, das der HC-SR04
Ultraschall-Entfernungssensor empfängt, wenn ein Hindernis 2 cm, 20 cm
und 400 cm entfernt ist, verwenden wir die Formel:

$\text{Dauer} = \frac{\text{Entfernung} \times 2}{\text{Schallgeschwindigkeit}}$

Die Schallgeschwindigkeit in Luft beträgt etwa 340 m/s (oder 0,034
cm/us). Da die Schallwelle die Entfernung zum Hindernis und zurück
zurückslegen muss, multiplizieren wir die Entfernung mit 2.

Beim Arbeiten mit dem HC-SR04 Ultraschallsensor und ähnlichen Sensoren
ist es üblich, die Zeit in Mikrosekunden (us) zu messen, da dies die
Genauigkeit der Entfernungsmessung erhöht.

\hypertarget{berechnung-fuxfcr-verschiedene-entfernungen}{%
\subsubsection{Berechnung für verschiedene
Entfernungen}\label{berechnung-fuer-verschiedene-entfernungen}}

\begin{enumerate}
\def\labelenumi{\arabic{enumi}.}

\item
  \textbf{Für ein Hindernis in 2 cm Entfernung:}
\end{enumerate}

$\text{Dauer} = \frac{2 \, \text{cm} \times 2}{0,034 \, \text{cm/us}} = \frac{4}{0,034} \, \text{us}$

\begin{enumerate}
\def\labelenumi{\arabic{enumi}.}
\setcounter{enumi}{1}

\item
  \textbf{Für ein Hindernis in 20 cm Entfernung:}
\end{enumerate}

$\text{Dauer} = \frac{20 \, \text{cm} \times 2}{0,034 \, \text{cm/us}} = \frac{40}{0,034} \, \text{us}$

\begin{enumerate}
\def\labelenumi{\arabic{enumi}.}
\setcounter{enumi}{2}

\item
  \textbf{Für ein Hindernis in 400 cm Entfernung:}
\end{enumerate}

$\text{Dauer} = \frac{400 \, \text{cm} \times 2}{0,034 \, \text{cm/us}} = \frac{800}{0,034} \, \text{us}$

\begin{lstlisting}[language=Python]
# Berechnung der Dauer des Echos für verschiedene Entfernungen in Mikrosekunden
schallgeschwindigkeit_cm_us = 0.034  # Schallgeschwindigkeit in cm/us

# Entfernungen in cm
entfernungen_cm = [2, 20, 400]

# Berechnung der Dauer für jede Entfernung in Mikrosekunden (us)
dauer_us = [(entfernung * 2 / schallgeschwindigkeit_cm_us) for entfernung in entfernungen_cm]

# Umrechnung der Dauer von Mikrosekunden in Millisekunden (ms)
dauer_ms = [dauer / 1000 for dauer in dauer_us]

# Umrechnung der Dauer von Millisekunden in Sekunden (s)
dauer_s = [dauer / 1000 for dauer in dauer_ms]

# Um ein besseres Verständnis für die Dauer zu bekommen (Berechnung des reziproken Werts der Dauer in Sekunden)
dauer_v = [1 / dauer if dauer != 0 else 0 for dauer in dauer_s]

# Ausgabe der Ergebnisse
dauer_us, dauer_ms, dauer_s, dauer_v
\end{lstlisting}

Die berechneten Ergebnisse für die Dauer des Echos bei verschiedenen
Entfernungen und deren Umrechnungen sind:

\begin{itemize}

\item
  \textbf{Dauer in Mikrosekunden (us):}

  \begin{itemize}
  
  \item
    2 cm Entfernung: 117.65 us
  \item
    20 cm Entfernung: 1176.47 us
  \item
    400 cm Entfernung: 23529.41 us
  \end{itemize}
\item
  \textbf{Dauer in Millisekunden (ms):}

  \begin{itemize}
  
  \item
    2 cm Entfernung: 0.118 ms
  \item
    20 cm Entfernung: 1.176 ms
  \item
    400 cm Entfernung: 23.529 ms
  \end{itemize}
\item
  \textbf{Dauer in Sekunden (s):}

  \begin{itemize}
  
  \item
    2 cm Entfernung: 0.000118 s
  \item
    20 cm Entfernung: 0.001176 s
  \item
    400 cm Entfernung: 0.023529 s
  \end{itemize}
\item
  \textbf{Reziproke Werte der Dauer in Sekunden (1/s, für ein besseres
  Verständnis der Dauer):}

  \begin{itemize}
  
  \item
    Bei einer \textbf{Entfernung von 2 cm} könnte der Schall das
    Hindernis und zurück zum Sensor \textbf{8500 Mal pro Sekunde}
    reisen.
  \item
    Bei einer \textbf{Entfernung von 20 cm} könnte dies \textbf{850 Mal
    pro Sekunde} geschehen.
  \item
    Und bei einer \textbf{Entfernung von 400 cm (4 Meter)} könnte der
    Schall diese Strecke \textbf{42,5 Mal pro Sekunde} zurücklegen.
  \end{itemize}
\end{itemize}

Diese Zahlen veranschaulichen die Häufigkeit, mit der der Schall die
angegebene Distanz innerhalb einer Sekunde hin und zurück reisen kann,
und bieten eine anschauliche Perspektive auf die Geschwindigkeit der
Schallausbreitung in Bezug auf die Entfernungsmessung mit dem HC-SR04
Ultraschallsensor.

\textbf{Um ein besseres Verständnis zu bekommen} eine Millisekunde ist
ein Tausendstel einer Sekunde, daher gilt:

$1 \text{ ms} = 0,001 \text{ s} = \frac{1}{1000}$

\begin{itemize}

\item
  \textbf{0,1 ms} entspricht $0,1 \times 0,001 = 0,0001$ etwa Sekunden
  \textbf{1/1000 einer Sekunde}
\item
  \textbf{1,1 ms} entspricht $1,1 \times 0,001 = 0,0011$ etwa Sekunden
  \textbf{1/10000 einer Sekunde}
\item
  \textbf{23,5 ms} entspricht $23,5 \times 0,001 = 0,0235$ etwa
  Sekunden \textbf{1/42 einer Sekunde}
\end{itemize}

\newpage

\hypertarget{programmierung-des-sensors}{%
\subsection{Programmierung des
Sensors}\label{programmierung-des-sensors}}

\begin{lstlisting}
// Informationsfluss für die Messung der Entfernung mit einem HC-SR04 Ultraschallsensor
+-----------------+      +-----------------+      +-----------------+
|   Start Setup   |      |   Initialisiere |      | Warte 4ms, dann |
| Serielle Komm.  | ---> | ULTRASCHALL_PIN | ---> | setze PIN als   |
|   beginnen      |      | als OUTPUT      |      | OUTPUT          |
+-----------------+      +-----------------+      +-----------------+
                                                         |
                                                         v
                                           +---------------------------+
                                           | Trigger-Pin auf LOW, dann |
                                           | warte 2us, sende HIGH-    |
                                           | Puls für 10us, setze      |
                                           | Trigger-Pin auf LOW       |
                                           +---------------------------+
                                                         |
                                                         v
                                         +-----------------------------+
                                         | Setze PIN als INPUT und     |
                                         | messe Dauer des Echos mit   |
                                         | pulseIn                     |
                                         +-----------------------------+
                                                         |
                                                         v
                                      +-----------------------------------+
                                      | Berechne Entfernung basierend auf |
                                      | der gemessenen Dauer und der      |
                                      | Schallgeschwindigkeit             |
                                      +-----------------------------------+
                                                         |
                                                         v
                                      +-----------------------------------+
                                      | Zeige gemessene Entfernung auf    |
                                      | dem seriellen Monitor an          |
                                      +-----------------------------------+
                                                         |
                                                         v
                                         +-----------------------------+
                                         | Warte 200ms vor der nächsten |
                                         | Messung                      |
                                         +-----------------------------+
                                                         |
                                                         v
                                             +-----------------+
                                             |  Wiederhole den |
                                             |  Messvorgang    |
                                             +-----------------+
Ausgabe: Die Entfernung beträgt: 10.13cm                                           
\end{lstlisting}

\newpage

\begin{lstlisting}[language={C++}]
/**
 * @file main.cpp
 * @brief Programmierung des Ultraschallsensors
 * 
 * Dieser Code ermöglicht es, die Entfernung zu einem Objekt mithilfe des HC-SR04 Ultraschallsensors zu
 * messen. Die serielle Kommunikation wird genutzt, um die gemessene Entfernung in Zentimetern auszugeben.
 * Der Sensor wird durch einen kurzen High-Puls aktiviert, und die Zeit, die das Echo benötigt, 
 * um zum Sensor zurückzukehren, wird verwendet, um die Entfernung zum Objekt zu berechnen.
 */
#include <Arduino.h>

// Definiere den Pin für das Ultraschallmodul
#define ULTRASONIC_PIN 10

void setup() {
  // Starte die serielle Kommunikation
  Serial.begin(115200);
}

void loop() {

  // Eine Verzögerung von 4ms ist erforderlich, sonst könnte das Ergebnis 0 sein
  delay(4);

  // Setze den Pin als Ausgang, um das Signal zu senden
  pinMode(ULTRASONIC_PIN, OUTPUT);

  // Setze den Trigger-Pin zurück auf niedrig
  digitalWrite(ULTRASONIC_PIN, LOW);
  delayMicroseconds(2);

  // Aktiviere den Sensor, indem ein High-Puls für 10us gesendet wird
  digitalWrite(ULTRASONIC_PIN, HIGH);
  delayMicroseconds(10);

  // Setze den Trigger-Pin wieder auf niedrig
  digitalWrite(ULTRASONIC_PIN, LOW);

  // Setze den Pin als Eingang, um zu lesen
  pinMode(ULTRASONIC_PIN, INPUT);

  // pulseIn gibt die Dauer des Pulses am Pin zurück
  float duration = pulseIn(ULTRASONIC_PIN, HIGH);

  // Berechne die Entfernung (in cm) basierend auf der Schallgeschwindigkeit (340 m/s oder 0,034 cm/us)
  float distance = duration * 0.034 / 2;

  // Zeige die Entfernung auf dem seriellen Monitor an
  Serial.print("Die Entfernung beträgt: ");
  Serial.print(distance);
  Serial.println(" cm");

  // Verzögere ein wenig, damit der Sensor sich vor der nächsten Messung stabilisieren kann
  delay(200);
}
\end{lstlisting}

\newpage

\hypertarget{programmierung-des-ultraschallmoduls-zur-steuerung-des-mars-rovers}{%
\subsection{Programmierung des Ultraschallmoduls zur Steuerung des
Mars-Rovers}\label{programmierung-des-ultraschallmoduls-zur-steuerung-des-mars-rovers}}

\textbf{Flussdiagramm} zeigt den Ablauf von der Messung der Entfernung
mit dem Ultraschallsensor bis zur Entscheidung, wie der Rover basierend
auf dieser Entfernung gesteuert werden soll. Es umfasst die
Initialisierung des Sensors, das Senden und Empfangen des
Ultraschallsignals, die Berechnung der Entfernung und die anschließende
Steuerung des Rovers je nach gemessener Entfernung.

\begin{lstlisting}
+----------------------+       +---------------------------+
|      Start loop()    |       |   readSensorData() wird   |
|                      | ----> |   aufgerufen               |
+----------------------+       +---------------------------+
                                       |
                                       v
                         +-------------------------------+
                         | Warte 4ms zur Stabilisierung  |
                         +-------------------------------+
                                       |
                                       v
                  +-----------------------------+  
                  | Trigger-Sequenz senden:     |
                  | 1. PIN auf OUTPUT           |
                  | 2. PIN auf LOW, 2us warten  |
                  | 3. PIN auf HIGH, 10us warten|
                  | 4. PIN auf LOW              |
                  +-----------------------------+
                                       |
                                       v
                     +-------------------------------+
                     | PIN auf INPUT setzen und      |
                     | Echo-Dauer mit pulseIn messen |
                     +-------------------------------+
                                       |
                                       v
            +--------------------------------------------+
            | Entfernung berechnen:                     |
            | duration * SCHALL_SPEED_CM_PER_US / 2     |
            +--------------------------------------------+
                                       |
                                       v
         +---------------------+          +----------------------+
         |  Ist Entfernung >   |   Nein   |  Ist Entfernung <    |
         |  SAFE_DISTANCE?     | -------- |  CLOSE_OBSTACLE_     |
         +---------------------+          |  DISTANCE und >      |
                | Ja                      |  MINIMUM_MEASURABLE_ |
                v                         |  DISTANCE?           |
    +----------------------+              +----------------------+
    | moveForward() mit    |                      | Ja
    | FORWARD_SPEED        |                      v
    +----------------------+             +------------------------+
                                        | moveBackward() mit     |
                                        | BACKWARD_SPEED, dann   |
                                        | warte 500ms             |
                                        +------------------------+
                                                    |
                                                    v
                                         +-------------------------+
                                         | backLeft() mit TURN_SPEED|
                                         | dann warte 1000ms       |
                                         +-------------------------+
                                                    |
                                                    v
                                          +-------------------------+
                                          |  Sonst: moveForward()   |
                                          |  mit VORSICHTIG_FORWARD_SPEED |
                                          +-------------------------+
\end{lstlisting}

\newpage

\begin{lstlisting}[language={C++}]
/**
 * @file main.cpp
 * @brief Programmierung des Ultraschallmoduls zur Steuerung des Mars-Rovers
 * 
 * Dieser Code nutzt den Ultraschallsensor, um die Entfernung zu Hindernissen zu messen 
 * und den Mars-Rover entsprechend zu steuern. 
 * Es wird entschieden, 
 * ob der Rover vorwärts, rückwärts oder in eine bestimmte Richtung bewegt werden soll, 
 * basierend auf der gemessenen Entfernung.
 */
#include <Arduino.h>
#include <SoftPWM.h>

// Definiere den Pin für das Ultraschallmodul
#define ULTRASCHALL_SENSOR_PIN 10

// Definition der Pins für die Motoren
#define LEFT_MOTOR_FORWARD_PIN 2
#define LEFT_MOTOR_REVERSE_PIN 3
#define RIGHT_MOTOR_FORWARD_PIN 5
#define RIGHT_MOTOR_REVERSE_PIN 4

// Konstanten für die Entfernungsmessung
const float SCHALL_SPEED_CM_PER_US = 0.034; // Schallgeschwindigkeit in cm/µs
const int SAFE_DISTANCE = 50;              // Sicherheitsabstand in cm
const int CLOSE_OBSTACLE_DISTANCE = 15;    // Abstand eines nahen Hindernisses in cm
const int MINIMUM_MEASURABLE_DISTANCE = 2; // Mindestmessbare Entfernung in cm
const int FORWARD_SPEED = 200;             // Geschwindigkeit für Vorwärtsbewegung
const int VORSICHTIG_FORWARD_SPEED = 70;    // Vorsichtige Vorwärtsbewegung
const int BACKWARD_SPEED = 70;            // Geschwindigkeit für Rückwärtsbewegung
const int TURN_SPEED = 70;                // Geschwindigkeit für Drehung

// Prototypen der Funktionen
float readSensorData(); // Liest die Daten vom Ultraschallsensor
void moveForward(int speed); // Bewegt den Rover vorwärts
void moveBackward(int speed); // Bewegt den Rover rückwärts
void backLeft(int speed); // Dreht den Rover nach links
void backRight(int speed); // Dreht den Rover nach rechts

void setup() {
  SoftPWMBegin(); // Initialisiere SoftPWM für Motorsteuerung
  Serial.begin(9600); // Starte die serielle Kommunikation zur Fehlersuche
}

void loop() {
  float distance = readSensorData();

  if (distance > SAFE_DISTANCE) { // Wenn der Weg frei ist, vorwärts bewegen
    moveForward(FORWARD_SPEED);
  } else if (distance < CLOSE_OBSTACLE_DISTANCE && distance > MINIMUM_MEASURABLE_DISTANCE) { // Wenn ein Hindernis nahe ist, aber über dem Mindestabstand, rückwärts bewegen
    moveBackward(BACKWARD_SPEED);
    delay(500); // Warte kurz, bevor versucht wird zu drehen
    backLeft(TURN_SPEED);
    delay(1000); // Wartezeit nach der Drehung
  } else { // Bei mittleren Distanzen vorsichtig vorwärts bewegen
    moveForward(VORSICHTIG_FORWARD_SPEED);
  }
}


// Funktionen-Implementierungen

// Liest die Entfernung vom Ultraschallsensor und berechnet sie
float readSensorData() {
  delay(4); // Wartezeit, um den Sensor zu stabilisieren

  // Trigger-Sequenz senden
  pinMode(ULTRASCHALL_SENSOR_PIN, OUTPUT);
  digitalWrite(ULTRASCHALL_SENSOR_PIN, LOW);
  delayMicroseconds(2);
  digitalWrite(ULTRASCHALL_SENSOR_PIN, HIGH);
  delayMicroseconds(10);
  digitalWrite(ULTRASCHALL_SENSOR_PIN, LOW);

  // Echo lesen
  pinMode(ULTRASCHALL_SENSOR_PIN, INPUT);
  float duration = pulseIn(ULTRASCHALL_SENSOR_PIN, HIGH);

  // Entfernung berechnen und zurückgeben
  return duration * SCHALL_SPEED_CM_PER_US / 2;
}

// Bewegt den Rover vorwärts
void moveForward(int speed) {
  SoftPWMSet(LEFT_MOTOR_FORWARD_PIN, speed);
  SoftPWMSet(LEFT_MOTOR_REVERSE_PIN, 0);
  SoftPWMSet(RIGHT_MOTOR_FORWARD_PIN, speed);
  SoftPWMSet(RIGHT_MOTOR_REVERSE_PIN, 0);
}

// Bewegt den Rover rückwärts
void moveBackward(int speed) {
  SoftPWMSet(LEFT_MOTOR_FORWARD_PIN, 0);
  SoftPWMSet(LEFT_MOTOR_REVERSE_PIN, speed);
  SoftPWMSet(RIGHT_MOTOR_FORWARD_PIN, 0);
  SoftPWMSet(RIGHT_MOTOR_REVERSE_PIN, speed);
}

// Dreht den Rover nach links
void backLeft(int speed) {
  SoftPWMSet(LEFT_MOTOR_FORWARD_PIN, 0);
  SoftPWMSet(LEFT_MOTOR_REVERSE_PIN, 0);
  SoftPWMSet(RIGHT_MOTOR_FORWARD_PIN, speed);
  SoftPWMSet(RIGHT_MOTOR_REVERSE_PIN, 0);
}

// Dreht den Rover nach rechts
void backRight(int speed) {
  SoftPWMSet(LEFT_MOTOR_FORWARD_PIN, speed);
  SoftPWMSet(LEFT_MOTOR_REVERSE_PIN, 0);
  SoftPWMSet(RIGHT_MOTOR_FORWARD_PIN, 0);
  SoftPWMSet(RIGHT_MOTOR_REVERSE_PIN, 0);
}
\end{lstlisting}

\hypertarget{reflexion-ultraschallsensor}{%
\subsection{Reflexion
Ultraschallsensor}\label{reflexion-ultraschallsensor}}

\begin{itemize}
\item
  Wie erkennt ein Infarotsensor Hindernisse? Können Sie das zugrunde
  liegende Konzept erläutern? (bei einem Wert von 0 wird ein Hindernis
  angenommen, und der Rover reagiert entsprechend, um Kollisionen zu
  vermeiden.)
\item
  Wie erkennt ein Ultraschallsensor Entfernungen? Können Sie das
  zugrunde liegende Konzept erläutern?
\item
  Wie unterscheidet sich das Hindernisvermeidungssystem mit
  Infarotsensor von dem Ultraschallsensor? Was sind ihre jeweiligen Vor-
  und Nachteile?
\item
  Ist es machbar, diese beiden Hindernisvermeidungssysteme
  (Infarotsensor und Ultraschallsensor) zu kombinieren?
\item
  \textbf{Erkennung von Hindernissen durch Infarotsensoren:}

  \begin{itemize}
  
  \item
    Infarotsensoren (IR-Sensoren) erkennen Hindernisse durch die
    Aussendung von Infrarotlicht und die Messung des reflektierten
    Lichts. Wenn ein Hindernis nahe genug ist, wird das ausgesendete
    Infrarotlicht zurück zum Sensor reflektiert. Bei einem Wert von 0
    wird angenommen, dass ein Hindernis vorhanden ist, was eine
    entsprechende Reaktion des Systems auslöst, um Kollisionen zu
    vermeiden.
  \end{itemize}
\item
  \textbf{Erkennung von Entfernungen durch Ultraschallsensoren:}

  \begin{itemize}
  
  \item
    Ultraschallsensoren messen Entfernungen durch die Aussendung von
    Ultraschallwellen. Diese Wellen treffen auf ein Objekt, werden
    reflektiert und zurück zum Sensor geschickt. Die Zeit, die vom
    Aussenden bis zum Empfangen des Echos vergeht, wird gemessen und
    genutzt, um die Entfernung zum Hindernis zu berechnen.
  \end{itemize}
\item
  \textbf{Unterschiede zwischen Infarot- und Ultraschallsensoren im
  Hindernisvermeidungssystem:}

  \begin{itemize}
  
  \item
    \textbf{Infarotsensoren:}

    \begin{itemize}
    
    \item
      Vorteile: Günstig, klein, niedriger Energieverbrauch.
    \item
      Nachteile: Beeinträchtigt durch externe Lichtquellen, begrenzte
      Reichweite, kann Schwierigkeiten haben, dunkle oder
      nicht-reflektierende Oberflächen zu erkennen.
    \end{itemize}
  \item
    \textbf{Ultraschallsensoren:}

    \begin{itemize}
    
    \item
      Vorteile: Funktioniert gut bei verschiedenen Lichtverhältnissen,
      kann größere Entfernungen messen, effektiv bei der Erkennung von
      Objekten unabhängig von deren Farbe und Oberflächenbeschaffenheit.
    \item
      Nachteile: Größer und teurer als IR-Sensoren, möglicherweise
      anfällig für Ultraschallechos in engen Räumen, die zu falschen
      Messungen führen können.
    \end{itemize}
  \end{itemize}
\item
  \textbf{Kombination von Infarot- und Ultraschallsensoren:}

  \begin{itemize}
  
  \item
    Die Kombination beider Sensorentypen in einem
    Hindernisvermeidungssystem ist machbar und kann die jeweiligen
    Schwächen ausgleichen. Durch die Nutzung beider Sensortypen kann ein
    robusteres System erstellt werden, das in einer Vielzahl von
    Umgebungen und unter verschiedenen Bedingungen effektiv
    funktioniert. Die Kombination ermöglicht eine präzisere und
    zuverlässigere Hinderniserkennung und -vermeidung.
  \end{itemize}
\end{itemize} % Platzhalter

%% Optional Anhang
%\clearpage
%\appendix

\clearpage
\printbibliography
\end{document}